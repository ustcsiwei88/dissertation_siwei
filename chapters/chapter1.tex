
\chapter{Introduction and Background} 
\thispagestyle{myheadings} 
In this chapter, we first discuss some regular types of sensing systems 
used for guarding, tracking or surveilence, 
with a focus on the line-based and range-based sensors studied in this dissertation. 
% and we then introduce the two types of simplified sensing model studied in this thesis, 
Then, we conduct a literature study of the related work on sensor placement and 
coverage related problems. 
Lastly, some background knowledge of the theories and tools used in this paper will be given. 
\section{Motivations} 
Sensor systems are ubiquitous. To list a few, systems of radar antennas 
or other sensor sources are frequently used as base stations for signal transmission, 
or intruder detection system (IDS) for monitoring hazards. 
Early intruder defense system can date back to ancient times, where 
watchtowers of the Great Wall of China are used as signal points. 
It can be seen as sensors from the broad sense 
since ancient soldiers lit woods to create smoke and inform others when invaders appear. 
% Surveilence or tracking cameras surveillance and tracking system, 
\begin{figure}[ht]
    \centering 
    \begin{subfigure}[b]{0.281\textwidth} 
        \includegraphics[width=\textwidth]{figures/Radar_antenna.jpg} 
        \caption{Radar antenna} 
        \label{fig:intro-radar} 
    \end{subfigure} 
    \begin{subfigure}[b]{0.48\textwidth} 
        \includegraphics[width=\textwidth]{figures/great_wall.jpg} 
        \caption{Watch towers on the great wall} 
        \label{fig:intro-great_wall} 
    \end{subfigure}
    \caption{Two examples of intrusion detection system}
    \label{fig:intro-IDS}
\end{figure} 

On top of autonomous vehicles, sensor systems are indispensible for obstacle avoidance and interactions between vehicles.
For example, Tesla (~\ref{fig:intro-tesla}) uses 12 ultrasonic sensors 
near the front and rear bumper 
and later changed into a vision system with only cameras 
\footnote{\url{https://www.tesla.com/en_eu/support/transitioning-tesla-vision}}. 
And TuSimple, an autonomous truck company, employs a combination of cameras, radars and lidars 
for their perception system \footnote{\\\url{https://www.tusimple.com/blogs/tusimple-1000-meter-perception-system}}. 

\begin{figure}[ht] 
    \centering 

    \begin{subfigure}[b]{0.49\textwidth} 
        \includegraphics[width=\textwidth]{figures/tesla.png} 
        \caption{
        Tesla model Y's sensing system
        equipped with cameras (\circled{\small{1}}, 
        \circled{\small{3}}, 
        \circled{\small{4}}, \circled{\small{5}}),
        ultrasonic sensors \circled{\small{2}}, and a radar \circled{6}.
        }
        \label{fig:intro-tesla} 
    \end{subfigure} \hfill
    \begin{subfigure}[b]{0.4\textwidth} 
        \includegraphics[width=\textwidth]{figures/tusimple.jpg} 
        \caption{TuSimple autonomous truck equipped with CMOS long-range cameras, 
        LiDARs and radars} 
        \label{fig:intro-truckcam} 
    \end{subfigure} 
    \caption{Sensor systems on autonomous vehicles}
    \label{fig:intro-autonomous-vehicles}
\end{figure} 

For surveilence or tracking systems, 
sensors like laser beams or cameras are deployed for applications like detecting thiefs, 
capturing motions, tracking poses and so on. 
\begin{figure}[ht] 
    \centering 
    
    % \begin{subfigure}[b]{0.55\textwidth} 
    %     \includegraphics[width=\textwidth]{figures/truck_cam.jpeg} 
    %     \caption{Autonomous truck camera system} 
    %     \label{fig:intro_truckcam} 
    % \end{subfigure} 
    \begin{subfigure}[b]{0.41\textwidth} 
        \includegraphics[width=\textwidth]{figures/optitrack.jpg} 
        \caption{Optical tracking system} 
        \label{fig:intro-optitrack} 
    \end{subfigure} 
    \hfill
    \begin{subfigure} [b]{0.46\textwidth} 
        \includegraphics[width=\textwidth]{figures/laser.jpg} 
        \caption{Laser system} 
        \label{fig:intro-laser} 
    \end{subfigure} 
\end{figure}

\section{Background}
In this section, we provide sufficient background knowledge and terms used in this dissertation,
they will be used without explaination in the following chapters. 

\subsection{NP and NP-hardness}
% In the area of optimization, which this thesis is focusing on, NP-hardness has the 
% The definitions are taken from \cite{vazirani2001approximation}
\begin{definition}[NP]
    A language $L\in NP$ if there is a polynomial $p$ and a polynomial time bounded Turing machine M, 
    called the {\textit verifier}, such that for each string $x\in \{0, 1\}^*$: 
    \begin{itemize}
        \item if $x\in L$, then there is a string $y$ (the certificate) of polynomially bounded length, i.e., $|y| \leq p(|x|)$,
        such that $M(x, y)$ accepts, and 
        \item if $x\notin L$, then for any string $y$, such that $|y|\leq p(|x|)$, $M(x,y)$ rejects.
    \end{itemize}
\end{definition}

A colloquial way in \cite{vazirani2001approximation} to describe an NP is the class of problems that have ``short and quickly verifiable''
Yes certificates.
% NP-complete problems refers to the class of problems such that every problem in NP can be reduced to it.
And NP-hard problem is the class of problems such that every problem in NP can reduce to it.
Typically, when we call an optimization problem NP-hard, it means the decision version of it is NP-hard.
\subsection{Integer programming}
Since most natural optimization problems are NP-hard, mathematical programming tools are often 
used for solving the problem for its generality and efficiency. Essentially, they take in
some mathematical models including a set of variables $x_1, \dots, x_n$ and a set of constraints,
\begin{align*}
    a_{11} x_{11} + \dots + & a_{1n} x_{1n} \geq b_1\\
    \dots & \\
    a_{m1} x_{m1} + \dots + & a_{mn} x_{mn} \geq b_m.
\end{align*}
% e.g., its usage in Multi-robot Path Planning Problem (MRPP) \cite{HanYu19IROS, GuoHanYu21ICRA}, sensor coverage.
Commercial integer programming tools include Gurobi \cite{optimization2019gurobi} and IBM CPLEX \cite{cplex2009v12}, while open-source 
libraries include SCIP \cite{achterberg2009scip}, CBC \cite{forrest2005cbc}, GLPK \cite{makhorin2008glpk}, and so on. 

\section{Literature review} 

\subsection{Coverage-related problems in computational geometry}

\subsection{Mobile sensing robot coverage control}

\subsection{Multi-robot coordination}

\subsection{Sensor network}
