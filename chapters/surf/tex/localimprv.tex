Whereas the ILP models for Problems~\ref{p:surf-1}-\ref{p:surf-3} support arbitrary precision, given that these problems are all computationally intractable, it can be expected that a pure ILP-based solution will only be scalable up to a certain point before an exorbitant amount computation time is needed. Inspired by the iterative update approach form \cite{cortes2004coverage}, we propose a two-phase optimization pipeline of 
using ILP (or the approximation algorithm for Problem~\ref{p:surf-2}) as the first phase with a good level of \emph{global} optimality guarantee and follow that with \emph{local} improvements that can be quickly computed to enhance the initial solution. We note that, as the local improvement is enhancing a solution with a level of global optimality guarantee, the enhancement is also global in effect. For example, in Problem~\ref{p:surf-2}, if we start with a $2$-approximation solution and obtain an initial coverage quality $r$ and subsequent local improvement reduce that to $0.75r$, then the final solution is a globally $1.5$-optimal solution.

We develop two such routines. The first is generally applicable and straightforward to implement: as the discretization level increases, we move the set of initial sensor locations (computed by Algorithm~\ref{alg:surf-greedy} or ILP) locally, one at a time. More formally, given an initial solution $\mathcal C = \{c_1, \dots, c_k\}$, denote $S_j \subset S$ as the region covered (possibly partially when working with Problem~\ref{p:surf-3}) by the sensor deployed at $c_j$. For each $S_j$, we try improving the quality of the solution by finding a better location for $c_j$ to cover $S_j$ at a finer resolution. Subsequently, $S_j$ can be updated based on the new $c_j$. The process may be repeated until convergence. 

The second local improvement routine is via solving a ``1-center'' like problem 
and is applicable to Problem~\ref{p:surf-2} and Problem~\ref{p:surf-3}. Due to limited space, 
we omit the lengthy algorithmic details and give a high-level description. 
For Problem~\ref{p:surf-2}, a sensor located at $c_j$ is ``responsible'' for visible points of $S$ 
that falls within a ball $B(c_j, r)$. 
Our improvement routine examines $S \cap B(c_j, r)$ and attempts to compute a new ball with 
a smaller radius that covers all of $S \cap B(c_j, r)$. 
The routine uses the ideas from Welzl's algorithm for computing minimum enclosing discs \cite{welzl1991smallest, Mark1997computation} and take time that is expected linear with respect to the number of samples that falls within $B(c_j, r)$, which is fairly fast. This method can be readily extended to Problem~\ref{p:surf-3} 
where the spheres become ``distorted'' (~\ref{fig:surf-exposure}).
