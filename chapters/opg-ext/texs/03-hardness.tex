We explore in this section the computational complexity of \opglr 
and \opgmc. Both problems are shown to be NP-hard with \opglr 
being strongly NP-hard. We later confirm that \opgmc is weakly 
NP-hard (in Section~\ref{sec:opgext-algorithm}).

\subsection{Strong NP-hardness of \opglr}\label{subsec:opgext-opglr-hardness}
When the number of types $t$ is a variable, i.e., $t$ is not a constant
and may be arbitrarily large,
% (but bounded by $n$), 
\opglr is shown to be NP-hard via the reduction from \tpart \cite{garey1975complexity}:

\vspace*{1mm}
\noindent
PROBLEM: \tpart\\
INSTANCE: A finite set A of $3m$ elements, a bound $B\in \mathbb{Z^+}$, 
% $S$ with $|S| = 3m$,
and a ``size'' $s(a)\in \mathbb{Z^+}$ for each $a\in A$,
such that each $s(a)$ satisfies $B/4 < s(a) <B/2$ and $\sum_{a\in A} s(a) = mB$.\\
QUESTION: Is there a partition of $S$ into $m$ disjoint subsets $S_1, 
\ldots, S_m$ such that for $1\leq i\leq m$, 
$\sum_{a\in S_i} s(a) = B$?
\vspace*{1mm}

%In a \tpart instance, let $\sum_{s \in S} s = B$ and for all $b \in B$, 
%$1/4 < b <  1/2$. Under such conditions, \tpart remains strongly 
%NP-hard \cite{garey1979computer}. We will use this version of the \tpart problem. We mention here that we use fractional values for 
%the elements of $B$ to avoid adding additional variables; this is fine 
%as long as we restrict the involved numbers to be rationals. 
\tpart is shown to be NP-complete in the strong sense\cite{GarJoh79}, 
i.e., it is NP-complete even when all numeric inputs are bounded by a polynomial 
of the input size. 

For the reduction, it is more convenient to work with a decision 
version of the \opglr problem, denoted as \opglrd. In the \opglrd 
problem, $a_{\tau}$ is the actual length robot type $\tau$ covers. 
That is, the coverage length of a robot is fixed. The \opglrd problem 
is specified as follows. 

\vspace*{1mm}
\noindent
PROBLEM: \opglrd\\
INSTANCE: $t$ types of robots where there are $n_{\tau}$ robots for 
each type $1 \le \tau \le t$; $n = n_1 + \cdots + n_t$. A robot of 
type $\tau$ has a coverage capacity $a_{\tau}$. A set of perimeters 
$\P = \{P_1, \ldots, P_m\}$ of a set of 2D regions 
$\R =\{R_1, \ldots, R_m\}$.\\ 
QUESTION: Is there a deployment of $n$ disjoint subsets $C_1, \ldots, C_n$
of $\{\partial R_1, \ldots, \partial R_m\}$ such that 
$P_1 \cup \ldots \cup P_m \subset C_1 \cup \ldots \cup C_n$, where
$C_j$ is a continuous segment for all $1 \le j \le n$, and for each 
$1 \le j \le n$, there is a unique robot whose type $\tau$, $1 \le \tau 
\le t$ satisfies $a_{\tau} \ge len(C_j)$?
\vspace*{1mm}

\begin{theorem}\label{t:opgext-opglr-hard}
	\opglr is strongly NP-hard. 
\end{theorem}
\begin{proof}
	A polynomial reduction from \tpart to \opglrd is constructed
	by a restriction of \opglrd. Given a \tpart instance with former notations,
	we apply several restrictions on \opglrd: {\em (i)} there are $3m$ types of robot
	and there is a single robot 
	for each type, i.e., $n_{\tau} = 1$ for $1 \le \tau \le t$, so $n=t=3m$ 
	{\em (ii)} the $3m$ capacities $a_1, \ldots, a_{3m}$ are set to be equal to
	$s(a)$ for each of the $3m$ elements $a\in A$, and {\em (iii)} 
	there are $3m$ perimeters and each perimeter $P_i$ is continuous and
	$len(P_i)=B$ for all $1 \le i \le m$.

	% \begin{comment}
	% In this proof, we work with a further restriction of \opglrd: {\em (i)}
	% there is a single robot for each type, i.e., $n_{\tau} = 1$ for $1 
	% \le \tau \le t$, and {\em (ii)} each perimeter $P_i$ is continuous and
	% %$len(P_i)$ is the same for all $1 \le i \le m$.
	
	% $len(P_i) = B$ for all $1 \le i \le m$.
	% Given a \tpart instance with $\sum_{b_i \in B} b = q$ and $1/4 \le b 
	% \le  1/2$ for all $b_i \in B$, let the corresponding \opglrd instance 
	% have $q$ perimeters, each with a single segment with the same length 
	% $L$. Let there be $3q$ robots, each with a type $1 \le \tau \le 3q$ 
	% and capability parameter $a_{\tau} = b_{\tau} \in B$. The restricted
	% \opglrd problem instance asks a question similar to the \tpart problem: 
	% is there a partition of the robots such that each perimeter of length 
	% $1$ is covered by three robots (notice that it is impossible to reach 
	% a total capacity of $1$ with fewer or more than $3$ robots)? 
	% \end{comment}

	% \begin{comment}
	% We make some rudimentary observations regarding the \opglrd instance 
	% and the related \opglr instance. Since all $q$ perimeters are of length 
	% $L$ each and the total capability of the robots is $q$, the objective 
	% \eqref{eq:objective} cannot be lower than $\frac{qL}{q} = L$, which is 
	% achieved only when each perimeter is covered by robots with total 
	% capability parameter of $1$. Then, because of the requirement $1/4 < 
	% a_{\tau} < 1/2$, if the best possible solution to the related \opglr 
	% instance is to be realized, each perimeter must be covered by exactly 
	% three robots; otherwise, the total capability cannot sum up to $1$. 
	% \end{comment}
	
	With the setup, the reduction proof is straightforward. Clearly, the 
	\tpart instance admits a partition of $A$ into $S_1, \ldots, 
	S_m$ such that $\sum_{a \in S_i} s(a) = B$ for all $1\leq i\leq m$ 
	if and only if a
	valid depolyment exists in the corresponding \opglrd instance. 
	It is clear that the reduction from \tpart to 
	\opglrd is polynomial (in fact, linear). Based on the 
	reduction and because \tpart is strongly NP-hard, so is \opglrd 
	%(by Lemma 4.1 of \cite{garey1975complexity}) 
	and \opglr.
\end{proof}

\begin{remark}
	One may also reduce weakly NP-hard problems, e.g., \twopart
	\cite{karp1972reducibility}, to \opglr for variable number of robot 
	types $t$. Being strongly NP-hard, \opglr is unlikely to admit pseudo-polynomial 
	time solutions for variable $t$. This contrasts with a later result 
	which provides a pseudo-polynomial time algorithm for \opglr for 
	constant $t$, as one might expect in practice where robots have limited 
	number of types. We also note that Theorem~\ref{t:opgext-opglr-hard} continues 
	to hold for a single perimeter with multiple segments, each 
	having a length $B$ in previous notation, separated by ``long'' gaps. Obviously, 
	\opglrd is in NP, thus rendering it NP-complete. 
\end{remark}

\subsection{NP-hardness of \opgmc}
The minimum cost \opg variant, \opgmc, is also NP-hard, which may be 
established through reduction from the \subsetsum problem 
\cite{karp1972reducibility}:

\vspace*{1mm}
\noindent
PROBLEM: \subsetsum \\
INSTANCE: A set $B$ with $|B| = n$ and a weight function $w: B \to 
\mathbb Z^+$, and an integer $W$.\\ 
QUESTION: Is there a subset $B' \subseteq B$ such that $\sum_{b \in B'} 
w(b) = W$?
\vspace*{1mm}

\begin{theorem}\label{t:opgext-opgmc-hard}
	\opgmc is NP-hard. 
\end{theorem}
\begin{proof}
	Given a \subsetsum instance, we construct an \opgmc instance with a 
	single perimeter containing a single segment with length $L$ to be 
	specified shortly. Let there be $t=2n$ types of robots. For $1 \le i 
	\le n$, let robot type $2i-1$ have $\ell_{2i-1} = c_{2i-1} = 
	w(b_i) + (2^{n + 1} + 2^i)W'$ and let robot type $2i$ have $\ell_{2i} 
	= c_{2i} = (2^{n + 1} + 2^i)W'$. Here, $W'$ can be any integer number no less than 
	$\sum_{b\in B} w(b)$. Set $L = W + (n2^{n+1} + 2^n + \ldots 
	+ 2^1)W'$. We ask the ``yes'' or ``no' decision question of whether there 
	are robots that can be allocated to have a total cost no more than $L$ 
	(equivalently, equal to $L$, as the cost density $c_\tau/l_\tau$
	is always $1$).
	
	Suppose the \subsetsum instance has a yes answer that uses a subset
	$B' \subseteq B$. Then, the \opgmc instance has a solution with cost $L$ 
	that can be constructed as follows. For each $1 \le i \le n$, a single 
	robot of type $2i - 1$ is taken if $b_i \in B'$. Otherwise, a single 
	robot of type $2i$ is taken. This allocation of robots yields a total 
	length and cost of $L$. 
	
	For the other direction, we first show that if the \opgmc instance 
	is to be satisfied, it can only use a single robot from type $2i-1$ 
	or $2i$ for all $1 \le i \le n$. First, if more than $n$ robots are 
	used, then the total cost exceeds $(n + 1)2^{n+1}W' > L$ as $W\leq W'$.
	% because $W' + (2^n + \ldots + 2)W' < 2^{n+1}W'$. 
	Similarly, if less than $n$ 
	robots are used, the total length is at most 
	$(n-1)2^{n+1}W'+(2^{n+1}-1)W'+ W' < L$. 
	Also, to match the $(2^n + \ldots + 2)W'$ part of the cost, exactly one robot 
	from type $2i-1$ or $2i$ for all $1 \le i \le n$ must be taken. 
	Now, if the \opgmc decision instance has a yes answer, if a robot 
	of type $2i -1$ is used, let $b_i \in B$ be part of $B'$, which 
	constructs a $B'$ that gives a yes answer to the \subsetsum instance.
\end{proof}
% add this citation because the proof the hardness probably comes from it
\begin{remark}
	It is also clear that the decision version of the \opgmc problem is 
	NP-complete. The \subsetsum is a weakly NP-hard problem that admits 
	a pseudo-polynomial time algorithm \cite{dantzig1957discrete}. 
	As it turns out, \opgmc, which shares similarities with \subsetsum
	and \ttkp (in particular, \ttukp \cite{ukphardness}), though NP-hard, 
	does admit a pseudo-polynomial time algorithm as well. 
\end{remark}

\begin{comment}
In concluding the section, we note that we have not determined the 
hardness for the case with constant number of robot types. There are 
some recent evidence that such problems may not be as hard as when 
the number of robot types is a fixed input parameter 
\cite{goemans2014polynomiality}.
\end{comment}

