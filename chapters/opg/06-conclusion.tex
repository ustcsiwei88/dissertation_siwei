In this chapter, we propose the \opg problem to model the allocation of 
large robotic swarms to cover complex 1D topological domains with 
optimality guarantees. For all variants under the \opg formulation 
umbrella, we have developed highly efficient algorithms for solving 
\opg exactly. In addition to rigorous proofs backed by formal analysis, 
extensive computational experiments further confirm the effectiveness of 
these algorithms. Moreover, practical relevance of \opg is demonstrated 
through the integration of \opg into realistic task (assignment) and motion 
planning scenarios. 

The study raises many additional interesting open questions; we mention 
a few here. 
%
First, the approach taken in this work is a {\em centralized} one where 
decision is made at the global level. It would be highly interesting to 
explore whether the same can be achieved with {\em decentralized} methods,
which have many advantages. For example, it may be the case that the 
gaps along the boundaries are not known {\em a priori} and must be measured
by the robots. In such cases, a centralized plan can be hard to come by. 
%
Second, as mentioned in Section~\ref{section:opg-problem}, the 
current \opg formulation assumes that the robots are confined to the 
boundaries $\partial \R$, which is one of many possible choices 
in terms of the robots' sensing and/or motion capabilities. In future study,
we plan to examine additional practical robot sensing and motion models. 
%
Third, as exact optimal algorithms are emphasized here, issues including 
uncertainty and robustness have not been touched in the current treatment, 
which are important elements when it comes to the deployment of a robotic 
swarm to tackle real-world challenges. 