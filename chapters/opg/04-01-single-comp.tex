When there is a single perimeter $P$, the solution is straightforward 
with $\ell^* = len(P)/n$. With $\ell^*$ determined, $\C^*$ is also readily computed. 

In the case where there are $m > 1$ regions, let the optimal 
distribution of the $n$ robots among the $m$ regions are given by 
$n_1^*, \ldots, n_m^*$. For a given region $R_i$, the $n_i^*$ robots 
must each guard a length $\ell_i = len(P_i)/n_i^*$. At this point, we
observe that for at least one region, say $R_i$, 
the corresponding $\ell_i$ must be maximal, i.e., $\ell_i = \ell^*$. 
The observation directly leads to a naive strategy for finding $\ell^*$: 
for each $R_i$, one may simply try all possible $1 \le n_i \le n$ and 
find the maximum $len(P_i)/n_i$ that is feasible, i.e., $n - n_i$ robots
can cover all other $R_{i'}$, $i' \ne i$, with each robot covering no 
more than $len(P_i)/n_i$. Denoting this $len(P_i)/n_i$ as $\ell_i^c$ and 
the corresponding $n_i$ as $n_i^c$, the smallest $\ell_i^c$ overall $1 
\le i \le m$ is then $\ell^*$.

The basic strategy mentioned above works and runs in polynomial time. 
However, it is possible to carry out the computation much more 
efficiently if the longest $P_i$ is examined first. Without loss of 
generality, assume that $P_1$ is the longest perimeter, i.e., $len(P_1)
\ge len(P_{i})$ for all $1 \le i \le m$. Recall that $n_1^c$ is the 
number of robots allocated to $P_1$ that yields $\ell_1^c$, it must hold 
that 
\begin{align}\label{eq:opg-l1}
\frac{len(P_1)}{n_1^c + 1} < \ell^* \le \frac{len(P_1)}{n_1^c} = \ell_1^c .
\end{align}
For an arbitrary $P_i$, simple manipulating of~\eqref{eq:opg-l1} yields
\begin{align}\label{eq:opg-li}
\frac{len(P_i)}{(n_1^c + 1)\frac{len(P_i)}{len(P_1)}} < \ell^* \le 
\frac{len(P_i)}{n_1^c\frac{len(P_i)}{len(P_1)}}.
\end{align}
This means that we only need to consider
$
n_i^c \in 
\big[\lceil n_1^c\frac{len(P_i)}{len(P_1)} \rceil, 
\lfloor (n_1^c + 1)\frac{len(P_i)}{len(P_1)}\rfloor].
$
However, since $P_1$ is the longest perimeter, $\frac{len(P_i)}{len(P_1)} 
\le 1$. Therefore, the difference between the two denominators 
of~\eqref{eq:opg-li} is no more than $1$, i.e., 
\[
(n_1^c + 1)\frac{len(P_i)}{len(P_1)} - n_1^c\frac{len(P_i)}{len(P_1)} \le 1. 
\]
When $len(P_i) \ne len(P_1)$, $(n_1^c + 1)\frac{len(P_i)}{len(P_1)} 
- n_1^c\frac{len(P_i)}{len(P_1)} < 1$ and there are two possibilities. One 
of these is 
$
\lceil n_1^c\frac{len(P_i)}{len(P_1)} \rceil =
\lfloor (n_1^c + 1)\frac{len(P_i)}{len(P_1)}\rfloor,
$
which leaves a single possible candidate for $n_i^c$. The other 
possibility 
is 
$
\lceil n_1^c\frac{len(P_i)}{len(P_1)} \rceil =
\lfloor (n_1^c + 1)\frac{len(P_i)}{len(P_1)}\rfloor + 1,
$
in which case there is actually no valid candidate for $n_i^c$. 
That is, after computing $n_1^c$ and $\ell_1^c$, if $len(P_i) = len(P_i)$ 
then no computation is needed for $P_i$. If $len(P_i) < len(P_1)$ then we 
only need to check at most one candidate for $n_i^c$. 

%The observation directly leads to an algorithm for computing $\ell^*$: 
%for each $P_i$, $1 \le i \le m$, we test all possible $1 \le n_i \le n 
%- m + 1$ (the rest of the $m -1$ regions need at least $m -1$ guards) 
%and for a fixed $n_i$ compute a candidate $\ell_i =  len(P_i)/n_i$. 
%This $\ell_i$ is assumed to be a potential $\ell^*$, which should be 
%the largest single robot coverage. Then, we check whether $n - n_i$ 
%robots can cover all perimeters except $P_i$ where each robot cannot 
%have coverage larger than $\ell_i$. For all feasible $1 \le n_i \le n$,
%the one that yields the smallest $\ell_i$ is kept. Denote this pair of 
%$n_i$ and $\ell_i$ as $n_i^{c}$ and $\ell_i^{c}$ where the superscript 
%means ``candidate''. Then, the smallest $\ell_i^{c}$ over all 
%$1 \le i \le m$ must be the desired $\ell^*$. 

Additional heuristics can be applied to reduce the required computation. 
First, in finding $n_1^c$, we may use bisection (binary search) over 
$[1, m]$ since if a given $n_1$ is infeasible, any $n_1' < n_1$ cannot 
be feasible either because $len(P_1)/n_1 < len(P_1)/n_1'$. Second, let 
$\ell = (\sum_{1\le i\le m}len(P_i))/n$, it holds that $\ell_i^c \ge 
\ell^* \ge \ell$. This means that for each $1 \le i \le m$, it is not 
necessary to try any $n_i > \lfloor \frac{len(P_i)}{\ell} \rfloor$. 
Third, if a candidate $\ell_i^c$ is at any time larger than the current 
candidate for $\ell^*$, that $i$ does not need to be checked further. 
We only use the first and the third in our implementation since the 
second does not help much once the bisection is applied. The 
pseudo code is outlined in Algorithm~\ref{algo:opg-MRS}. Note that we 
assume the problem instance is feasible ($n \ge m$), which is easy to check. 

It is straightforward to verify that Algorithm~\ref{algo:opg-MRS} runs in 
time $O(m\log n + m^2)$. The $O(m\log n)$ comes from the $\mathbf{while}$ 
loop, which calls the function \isLFeasible($\ell_i^c$, $n_i^c$, $i$) 
$\log n$ time. The function checks whether the current $\ell_i^c$ is 
feasible for perimeters other than $P_i$ (note that it is assumed that 
\isLFeasible($\cdot$) has access to the input to Algorithm~\ref{algo:opg-MRS} 
as well). This is done by computing for $i' \ne i$, $n_{i'} = \lceil 
len(P_{i'})/\ell_i^c \rceil$ and checking whether $\sum_{{i'} \ne i}n_{i'} 
\le n - n_i^c$. The $O(m^2)$ term comes from the $\mathbf{for}$ loop. 
The running time of Algorithm~\ref{algo:opg-MRS} may be further reduced by 
noting that the $\mathbf{for}$ loop examines $(m-1)$ candidate 
$\ell_i^c$. These $\ell_i^c$ can be first computed and sorted, on which 
bisection can be applied. This drops the main running time to 
$O(m(\log n + \log m))$. This second bisection is not reflected in 
Algorithm~\ref{algo:opg-MRS} to keep the logic and notation more 
straightforward. If we also consider input complexity, an additional 
$O(\sum_{1\le i \le m} M_i)$ is needed to compute $len(P_i)$ from the raw polygonal 
input and an additional $O(n)$ time is needed for generating the actual
locations for the $n$ robots. The total complexity is then $O(m(\log n + 
\log m) + \sum_{1\le i \le m} M_i + n)$.
%\sh{Algorithm style: 
%(1) Change displaystyle to remove small font.
%(2) Tighten up by using lIf instead of If.}
%\jy{We can do some of these if needed. I used this style for some 
%particular reason that I can tell you :).}
% \vspace*{-3mm}

\begin{algorithm}[H]
	\SetKwInOut{Input}{Input}
	\SetKwInOut{Output}{Output}
	\SetKwComment{Comment}{\%}{}
	\setstretch{0.8}
\begin{small}
	\Input{$P_1, \ldots, P_m$: each $P_i$ a continuous polygonal line; 
		assume that $P_1$ is a longest perimeter \\
		$n$: the number of robots}
	\Output{$\ell^*, i^*$: the optimal coverage and the $i$ realizing it}
	\vspace{0.025in}

	$\ell^* \leftarrow \infty$; $i^* \leftarrow 1$;
	\vspace{0.025in}

	\Comment{\footnotesize Compute $n_1^c$ and initial $\ell^*$.}

	$n_1^{min} \leftarrow 1$; $n_1^{max} \leftarrow n$; $n_1^c \leftarrow 1$;
	\vspace{0.025in}

	\While{$n_1^{min} \ne n_1^{max}$}{
	\vspace{0.025in}
	$n_1 \leftarrow \lceil \frac{n_1^{min} + n_1^{max}}{2} \rceil$;
	$\ell_1 \leftarrow \frac{len(P_1)}{n_1}$;

	\vspace{0.025in}
	\uIf{\isLFeasible($\ell_1$, $n_1$, $1$)}{
		\vspace{0.025in}
		$\ell^* \leftarrow \ell_1$; 
		$n_1^c \leftarrow n_1$;
		$n_1^{min} \leftarrow n_1$; 
	}
	\Else{
		$n_1^{max} \leftarrow n_1 - 1$; 
	}
	}

	\vspace{0.05in}
	\Comment{\footnotesize Optimize $\ell^*$ over all $2 \le i \le m$.}


	\For{$i \in \{2,\dots, m\}$ }{\label{algo:opg-mrs:for}
		$n_i^- = \lceil \frac{n_1^clen(P_i)}{len(P_1)} \rceil$;
		$n_i^+ = \lfloor \frac{(n_1^c +1)len(P_i)}{len(P_1)} \rfloor$; 
		$\ell_i \leftarrow \frac{len(P_i)}{n_i^+}$;
					
			\If{$n_i^- == n_i^+$ and \isLFeasible($\ell_i$, $n_i^+$, $i$) and $\ell_i < \ell^*$}{
				\vspace{0.025in}
				$\ell^* \leftarrow \ell_i$; $i^* \leftarrow i$;
			}
	}
	\Return{$\ell^*$, $i^*$}
\end{small}
	\caption{\algoMRSimple} \label{algo:opg-MRS}
\end{algorithm}
% \vspace*{-6mm}
