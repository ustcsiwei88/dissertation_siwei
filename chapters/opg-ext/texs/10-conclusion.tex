In this paper, we investigate two natural models of optimal perimeter 
guarding using heterogeneous robots, where one model (\opglr) limits 
the number of available robots and the second (\opgmc) seeks to 
optimize the total cost of coverage. 

These formulations have many potential applications. One application 
scenario we envision is the deployment of multiple agents or robots 
as ``emergency responders'' that are constrained to travel on the 
boundary. An optimal coverage solution will then translate to minimizing 
the maximum response time anywhere on the perimeter (the part that 
needs guarding). The scenario applies to \opg, \opglr, and \opgmc. 

Another application scenario is the monitoring of the perimeter 
using robots with different sensing capabilities. A simple heterogeneous 
sensing model here would be robots equipped with cameras with different 
resolutions, which may also be approximated as discs of different radii. 
The model makes sense provided that the region to be covered is much 
larger than the sensing range of individual robots and assuming that the 
boundary has relatively small curvature as compared to the inverse of the 
radius of the smallest sensing disc of the robots. For boundary with 
relatively small curvature, our solutions would apply well to the sensing 
model by using the diameter of the sensing disc as the 1D sensing range. 
As the region to be covered is large, covering the boundary will require
much fewer sensors than covering the interior. 


On the computational complexity 
side, we prove that both \opglr and \opgmc are NP-hard, with \opglr 
directly shown to be strongly NP-hard. This is in stark contrast to 
the homogeneous case, which admits highly efficient low polynomial 
time solutions \cite{FenHanGaoYu19RSS}. The complexity study also 
establishes structural similarities between these problems and 
classical NP-hard problems including \tpart, \ttkp, and \subsetsum.

On the algorithmic side, we provide methods for solving both \opglr 
and \opgmc exactly. For \opglr, the algorithm runs in pseudo-polynomial 
time in practical settings with limited types of robots. In 
this case, the approach is shown to be computationally effective. 
For \opgmc, a pseudo-polynomial time algorithm is derived for the 
general problem, which implies that \opgmc is weakly NP-hard. In 
practice, this allows us to solve large instances of \opgmc. We 
further show that a polynomial time algorithm is possible for 
\opgmc when the types of robots are fixed. 

With the study of \opg \cite{FenHanGaoYu19RSS} for homogeneous and 
heterogeneous cases, some preliminary understanding has been 
obtained on how to approach complex 1D guarding problems. 
Nevertheless, the study so far is limited to {\em one-shot} settings
where the perimeters do not change. In future research, we would like 
to explore the more challenging case where the perimeters evolve 
over time, which requires the solution to be dynamic as well. Given 
the results on the one-shot settings, we expect the dynamic setting
to be generally intractable if global optimal solutions are desired, 
potentially calling for iterative and/or approximate solutions. 

We recognize that our work 
does not readily apply to a visibility-based sensing model, which is also 
of interest. Currently, we are also exploring covering of the interior
using range-based sensing. As with the OPG work, we want to push for 
optimal or near-optimal solutions when possible.
\vspace*{-1mm}
