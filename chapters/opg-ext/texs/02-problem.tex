Let $\W \subset \mathbb R^2$ be a compact (closed and bounded) 
two-dimensional workspace. There are  $m$ pairwise disjoint {\em 
	regions} $\R = \{R_1, \ldots, R_m\}$ where each region $R_i \subset \W$ 
is homeomorphic to the closed unit disc, i.e., there exists a continuous 
bijection $f_i: R_i \to \{(x, y) \mid x^2 + y^2 \le 1\}$ for all $1 \le 
i \le m$. For a given region $R_i$, let $\partial R_i$ be its (closed) 
boundary (therefore, $f_i$ maps $\partial R_i$ to the unit circle  
$\mathbb S^1$). With a slight abuse of notation, define $\partial \R 
= \{\partial R_1, \ldots, \partial R_m\}$. Let $P_i \subset \partial R_i$ 
be the part of $\partial R_i$ that is accessible, specifially, not blocked by 
obstacles in $\W$. This means that each $P_i$ is either a single closed 
curve or formed by a finite number of (possibly curved) line segments. 
Define  $\P = \{P_1, \ldots, P_m\} \subset \W$ as the {\em perimeter} 
of $\R$ which must be {\em guarded}. More formally, each $P_i$ is 
homeomorphic to a compact subset of the unit circle (i.e., it is 
assumed that the maximal connected components of $P_i$ are closed 
line segments). For a given $P_i$, each one of its maximal connected 
component is called a {\em perimeter segment} or simply a {\em segment}, 
whereas each maximal connected component of $\partial R_i \backslash P_i$ 
is called a {\em perimeter gap} or simply a {\em gap}. An example setting is 
illustrated in ~\ref{fig:opgext-example-boundaries} with two regions. 
\definecolor{BrickRed}{RGB}{176, 50, 28}
\definecolor{ForestGreen}{RGB}{0, 155, 85}
\begin{figure}[ht]
	\begin{center}
		\begin{overpic}[width=0.7\textwidth,tics=5]
			{chapters/opg-ext/figures/example-boundaries-eps-converted-to.pdf}
			\put(26,20){{\small $R_1$}}
			\put(20,39){{\small \textcolor{BrickRed}{$P_1$}}}
			\put(66,28){{\small $R_2$}}
			\put(54,40){{\small \textcolor{ForestGreen}{$P_2$}}}
			%\put(79.6,18){{\small $h$}}
			\put(82,44){{\small $\W$}}
		\end{overpic}
	\end{center}
	\caption{\label{fig:opgext-example-boundaries} An example of a workspace $\W$ 
		with two regions $\{R_1, R_2\}$. Due to three {\em gaps} on $\partial R_1$, 
		marked as dotted lines within long rectangles, $P_1 \subset \partial R_1$ 
		has three {\em segments} (or maximal connected components); $P_2 = \partial 
		R_2$ has a single segment with no gap.}
\end{figure}

After deployment, some number of robots are to {\em cover} the perimeter 
$\P$ such that a robot $j$ is assigned a continuous closed subset $C_j$ 
of some $\partial R_i, 1 \le i \le m$. All of $\P$ must be {\em covered} 
by $\C$, i.e., 
%
$\bigcup_{P_i \in \P} P_i  \subset \bigcup_{C_j \in \C} C_j$,
%
which implies that elements of $\C$ need not intersect on their interiors. 
Hence, it is assumed that any two elements of $\C$ may share at most their 
endpoints. Such a $\C$ is called a {\em cover} of $\P$. Given a cover 
$\C$, for a $C_j \in \C$, let $len(C_j)$ denote its length (more formally, 
measure). 

To model heterogeneity of the robots, two models are explored in this
study. In either model, there are $t$ types of robots. In the first model,
the number of robots of each type is fixed to be $n_1, \ldots, n_t$ with 
$n = n_1 + \cdots + n_t$. For a robot $1 \le j \le n$, let $\tau_j$ denote 
its type. Each $1 \le \tau \le t$ type of robots has some 
level of {\em capability} or {\em ability} $a_{\tau} \in \mathbb Z^+$. We 
wish to balance the load among all robots based on their capabilities, 
i.e., the goal is to find cover $\C$ for all robots such that the quantity 
\[
\max_{C_j \in \C} \frac{len(C_j)}{a_{\tau_j}},
\]
which represents the largest coverage-capacity ratio, is minimized. 
We note that when all capacities are the same, e.g., $a_{\tau} = 1$ for 
all robots, this becomes the standard \opg problem studied in \cite{fenghangaoyu2019efficient}. 
We call this version of the perimeter guarding problem {\em optimal 
	perimeter guarding with limited resources} or \opglr. The formal 
definition is as follows.

\begin{problem}[Optimal Perimeter Guarding with Limited Resources 
	(\opglr)] Let there be $t$ types of robots. For each type $1\le \tau 
	\le t$, there are $n_{\tau}$ such robots, each having the same 
	capability parameter $a_{\tau}$. Let $n = n_1 + \cdots + n_t$. 
	Given the perimeter set $\P = \{P_1, \ldots, P_m\}$ of a set of 
	2D regions $\R =\{R_1, \ldots, R_m\}$, find a set of $n$ continuous 
	line segments $\C^* = \{C_1^*, \ldots, C_n^*\}$ such that $\C^*$ covers 
	$\P$, i.e., \begin{align}\label{eq:opgext-opg-coverage}
	\bigcup_{P_i \in \P} P_i  \subset \bigcup_{C_j^* \in \C^*} C_j^*,
	\end{align}
	such that a $C_j^*$ is covered by robot $j$ of type $\tau_j$, and such that,
	among all covers $\C$ satisfying~\eqref{eq:opgext-opg-coverage}, 
	\begin{align}\label{eq:opgext-opg-objective}
	\C^* = \underset{\C}{\mathrm{argmin}} \max_{C_j \in \C} 
	\frac{len(C_j)}{a_{\tau_j}}.
	\end{align}
\end{problem}

Whereas the first model caps the number of robots, the second
model fixes the maximum coverage of each type of robot. That is, for 
each robot type $1 \le \tau \le t$, $n_{\tau}$, the number of robots of type $\tau$,
is unlimited as long as it is non-negative, but each such robot can only cover 
a maximum length of $\ell_{\tau}$. 
At the same time, using each such robot incurs a cost of $c_{\tau}$. The 
goal here is to guard the perimeters with the minimum total cost. We 
denote this problem {\em optimal perimeter guarding with minimum 
	cost} or \opgmc. 

\begin{problem}[Optimal Perimeter Guarding with Minimum Cost
	(\opgmc)] Let there be $t$ types of robots of unlimited quantities. 
	For each robot of type $1\le \tau \le t$, it can guard a length of 
	$\ell_{\tau}\in\mathbb{Z^+}$ with a cost of $c_{\tau}\in\mathbb{Z^+}$. Given the perimeter set
	$\P = \{P_1, \ldots, P_m\}$ of a set of 2D regions $\R =\{R_1, \ldots, 
	R_m\}$, find a set of $n = n_1 + \cdots + n_t$ continuous line segments 
	$\C^* = \{C_1^*, \ldots, C_n^*\}$ where $n_{\tau}$ such segments are 
	guarded by type $\tau$ robots, such that $\C^*$ covers $\P$, i.e., 
	\begin{align}\label{eq:opgext-opg-coverage2}
	\bigcup_{P_i \in \P} P_i  \subset \bigcup_{C_j^* \in \C^*} C_j^*,
	\end{align}
	such that a $C_j^*$ is covered by robot $j$ of type $\tau_j$, i.e., 
	$C_j^* \le \ell_{\tau_j}$, and such that,
	among all covers $\C$ satisfying~\eqref{eq:opgext-opg-coverage2}, 
	\begin{align}\label{eq:opgext-opg-objective2}
	\C^* = \underset{\C}{\mathrm{argmin}} \sum_{1 \le \tau \le t} 
	n_{\tau}c_{\tau}.
	\end{align}
\end{problem}
