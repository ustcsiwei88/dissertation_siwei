
\chapter{Conclusion \& Future Work}
\thispagestyle{myheadings}
In this thesis, we studied the optimal multi-sensor placement problems with different sensing models:
distance-based line sensor, line-of-sight line sensor, range sensor, mobile line sensor, and mobile point sensing robot.
Some of the problems are solvable in low polynomial time with traditional methods, while most of them are NP-hard.
Our way to duel with these intractable problems includes methods like dynamic programming, approximation algorithms,
local search-type method, integer programming and their combinations. 

In the future, the work in this thesis can be extended in multiple directions.

First, the thesis focus on sensing range as some basic shapes like lines or circles, and they are considered individually.
A heterogeneous combination of sensors like line sensors used together with range sensors can be very interesting to explore.
Another possibility is to model the uncertainty of sensors like sensor blackout, which is considered in works like \cite{Olsen2022ICRA}.
All of these sensing models are used to simplify the sophisticatedness of real-world sensing models of laser beams, radars, lidars and so on.
So, any model that can drag the problem closer to reality should be valuable to explore.

Second, more realistic constraints related to mobile robots can be added to the problem of sweep line coverage or boundary defense: 
e.g., for the mobile sensing robot, dynamic constraints like acceleration and turning radius of the robot are not considered in our work, 
as well as the shape of the robot. 
Considering these will increase the effectiveness of applying the algorithms developed on real-world mobile sensing robot applications.
Also, for a mobile sensing robot, other constraints can be modeled such as energy consumption (where acceleration and deceleration matter), and
blockage of view from each other.

% Third, a missing part in our work 