
Let $\W \subset \mathbb R^2$ be a compact (i.e., closed and bounded) 
two-dimensional workspace. There are  $m$ pairwise disjoint {\em 
regions} $\R = \{R_1, \ldots, R_m\}$ where each region $R_i \subset \W$ 
is homeomorphic to the closed unit disc, i.e., there exists a continuous 
bijection $f_i: R_i \to \{(x, y) \mid x^2 + y^2 \le 1\}$ for all $1 \le 
i \le m$. For a given region $R_i$, let $\partial R_i$ be its (closed) 
boundary (therefore, $f_i$ maps $\partial R_i$ to the unit circle  
$\mathbb S^1$). With a slight abuse of notation, define $\partial \R 
= \{\partial R_1, \ldots, \partial R_m\}$. Let $P_i \subset \partial R_i$ 
be the part of $\partial R_i$ that is accessible, e.g., not blocked by 
obstacles in $\W$. This means that each $P_i$ is either a single closed 
curve or formed by a finite number of (possibly curved) line segments. 
Define  $\P = \{P_1, \ldots, P_m\} \subset \W$ as the {\em perimeter} 
of $\R$ which must be {\em guarded}. More formally, each $P_i$ is 
homeomorphic to a compact subset of the unit circle (i.e., it is 
assumed that the maximal connected components of $P_i$ are closed 
line segments). For a given $P_i$, each one of its maximal connected 
component is called a {\em perimeter segment} or simply a {\em segment}, 
whereas each maximal connected components of $\partial R_i \backslash P_i$ 
is called a {\em perimeter gap} or simply a {\em gap}. An example setting is 
illustrated in Fig.~\ref{fig:example-boundaries} with two regions. 

\begin{figure}[ht]
\vspace*{-1mm}
\begin{center}
\begin{overpic}[width=0.7\textwidth,tics=5]
{./chapters/opg/figures/example-boundaries.eps}
\put(26,20){{\small $R_1$}}
\put(20,39){{\small \textcolor{red}{$P_1$}}}
\put(66,28){{\small $R_2$}}
\put(54,40){{\small \textcolor{green}{$P_2$}}}
%\put(79.6,18){{\small $h$}}
\put(82,44){{\small $\W$}}
\end{overpic}
\end{center}
\vspace*{-4mm}
\caption{\label{fig:opg-example-boundaries} An example of a workspace $\W$ 
with two regions $\{R_1, R_2\}$. Due to three {\em gaps} on $\partial R_1$, 
marked as dotted lines within long rectangles, $P_1 \subset \partial R_1$ 
has three {\em segments} (or maximal connected components); $P_2 = \partial 
R_2$ has a single segment with no gap.}
\vspace*{-3mm}
\end{figure}

There are $n$ indistinguishable point robots residing in $\W$.
%and outside of any $R_i \in \R$, $1 \le i \le m$. 
These robots are to be deployed to {\em cover} the perimeter $\P$ such 
that each robot $1 \le j \le n$ is assigned a continuous closed subset 
$C_j$ of some $\partial R_i, 1 \le i \le m$. All of $\P$ must be 
{\em covered} by $\C$, i.e., 
%
$\bigcup_{P_i \in \P} P_i  \subset \bigcup_{C_j \in \C} C_j$,
%
which implies that elements of $\C$ need not intersect on their
interiors. Hence, it is assumed that any two elements of $\C$ may share 
at most their endpoints. Such a $\C$ is called a cover of $\P$. 

Given a cover $\C$, for a $C_j \in \C$, $1 \le j \le n$, let $len(C_j)$ 
denote its length (more formally, measure). It is desirable to minimize 
the maximum $len(C_j)$, i.e., the goal is to find a cover $\C$ such that 
the value 
%
$\max_{C_j \in \C} len(C_j)$
%
is minimized. This corresponds to minimizing the maximum workload for 
each robot or agent. The formal definition of the Optimal Perimeter 
Guarding (\opg) problem is provided as follows. 

\begin{problem}[Optimal Perimeter Guarding (\opg)] Given the perimeter 
$\P = \{P_1, \ldots, P_m\}$ of a set of 2D regions $\R =\{R_1, \ldots, 
R_m\}$, find a set of $n$ continuous line segments $\C^* = \{C_1^*, 
\ldots, C_n^*\}$ such that $\C^*$ covers $\P$, i.e., 
\begin{align}\label{eq:opg-coverage}
\bigcup_{P_i \in \P} P_i  \subset \bigcup_{C_j^* \in \C^*} C_j^*,
\end{align}
with the maximum of $len(C_j^*), 1 \le j \le n$ minimized, i.e., among all 
covers $\C$ satisfying~\eqref{eq:coverage}, 
\begin{align}\label{eq:opg-objective}
\C^* = \underset{\C}{\mathrm{argmin}} \max_{C_j \in \C} len(C_j).
\end{align}
\end{problem}

Here, we introduce the technical assumption that the ratio between 
the length of $\partial \R$ and the length of $\partial \P$ is polynomial
in the input parameters. That is, the length of $\partial \R$ is not much 
bigger than the length of $\partial \P$ . 
The assumption makes intuitive sense as any 
gap should not be much bigger than the perimeter in pratice. We note 
that the assumption is not strictly necessary but helps simplify the 
correctness proof of some algorithms. 

Henceforth, in general, $\C^*$ is used when an optimal cover is meant 
whereas $\C$ is used when a cover is meant. We further define the optimal 
single robot coverage length as 
\begin{align}\label{eq:opg-l-star}
\vspace*{-1mm}
\ell^* = \min_{\C} \max_{C_j \in \C} len(C_j).
\vspace*{-1mm}
\end{align}

Fig.~\ref{fig:opg-example} shows an example of an optimal cover by $8$ robots 
of a perimeter with three components. Note that one of the three gaps 
(the one on the top area as part of the hexagon) is fully covered 
by a robot, which leads to a smaller $\ell^*$ as compared to other 
feasible solutions. This interesting phenomenon, which is 
actually a main source of the difficulty in solving \opg, is explored 
more formally in Section~\ref{section:opg-analysis} 
(Proposition~\ref{p:opg-max-no-exclusion}).

Given the \opg formulation, additional details on $\partial \R$ must 
be specified to allow the precise characterization of the computational 
complexity (of any algorithm developed for \opg). For this purpose, it 
is assumed that each $\partial R_i \in \partial \R$, $1 \le i \le m$, 
is a simple (i.e., non-intersecting and without holes) polygon with an 
input complexity $O(M_i)$, i.e., $\partial R_i$ has about $M_i$ vertices 
or edges. If an \opg has a single region $R$, then let $\partial R$ have 
an input complexity of $M$. Note that the algorithms developed in this 
work apply to curved boundaries equally well, provided that the curves 
have similar input complexity and are given in a format that allow the 
computation of their lengths with the same complexity. 
Alternatively, curved boundaries may be approximated to arbitrary 
precision with polygons. 

%\jy{Add an example showing an optimal cover for maybe some $6-10$ robots?
%This can probably go in the introduction to motivate the paper. We need 
%some 1-2 good examples. Need to refer back to the example here.}

For deploying a robot to guard a $C_j$, one natural choice is to send the 
robot to a location $t_j \in C_j$ such that $t_j$ is the centroid of $C_j$. 
Since $C_j$ is one dimensional, $t_j$ is the center (or midpoint) of $C_j$. 
After solving an \opg, there is the remaining problem of assigning the $n$ 
robots to the centers of $\C^* = \{C_j^*\}$ and actually moving the robots 
to these assigned locations. As a secondary objective, it may also be 
desirable to provide guarantees on the execution time required for 
deploying the robots to reach target guarding locations. We note that, 
the task assignment (after determining target locations) and motion 
planning component for handling robot deployment, essential for applications 
but not a key part of this work's contribution, is briefly addressed in 
Section~\ref{section:opg-evaluation}. 

With some $\C^*$ satisfying~\eqref{eq:opg-coverage} 
and~\eqref{eq:objective}, we may further require that $len(C_j^*)$ is 
minimized for all $C_j^* \in \C^*$. This means that a gap $G \subset 
((\bigcup \partial R_i)\backslash (\bigcup P_i))$ will never be partially 
covered by some $C_j^* \in \C^*$ because if that is the case, $C_j^*$ 
needs not cover any part of $G$ at all (and should not, to limit the 
coverage of the assigned robot). In the example from 
Fig.~\ref{fig:opg-example-boundaries}, $G$ may be one of the three dotted 
lines on $\partial R_1$; clearly, it is not beneficial to have some 
$C_j^*$ partially cover (i.e., intersect the interior of) one of these. 
This rather useful condition (note that this is not an assumption but 
a solution property) yields the following lemma. 

\begin{lemma}\label{l:opg-no-partial-coverage} 
For a set of perimeters $\P = \{P_1,\ldots, P_m\}$ where $P_i \subset 
\partial R_i$ for $1 \le i \le m$, there exists an optimal cover $\C^* 
= \{C_1^*, \ldots, C_n^*\}$ such that, for any gap (or maximal connected
component) $G \subset ((\bigcup \partial R_i)\backslash (\bigcup P_i))$ 
and any $C_j^* \in \C^*$, $C_j^* \cap G = G$ or $C_j^* \cap G = 
\varnothing$. 
\end{lemma}

\begin{remark} Our definition of coverage is but one of the possible 
models of coverage. The definition restricts a robot deployed to 
$C_j, 1 \le j \le n$, to essential {\em live} on $C_j$. The definition 
models scenarios where a guarding robot must travel along $C_j$, which 
is one-dimensional. Nevertheless, the algorithms developed for \opg have 
broader applications. For example, subroutines in our algorithms readily 
solve the problem of finding the minimum number of guards needed if each 
guard has a predetermined maximum coverage. 
%As we will demonstrate later, by selecting the 
%boundaries (i.e., $\{\partial R_i\}$) carefully, the model applies 
%to diverse practical application scenarios.
\end{remark}
