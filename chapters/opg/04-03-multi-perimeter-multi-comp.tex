The algorithm for the multiple perimeter case is a direct 
generalization Algorithm~\ref{algo:opg-SRG}. To facilitate the description, 
let the perimeter $P_i$, $1 \le i \le m$, contain $q_i$ maximal connected 
components, i.e., $P_i = S_{i,1} \cup\ldots \cup S_{i,q_i}$ and the 
boundary $\partial R_i = S_{i,1} \cup G_{i,1} \cup \ldots \cup S_{i,q_i} 
\cup G_{i,q_i}$. We extend the definition of $S_{k-k'}$ for a single 
perimeter to $S_{i,k-k'}$ for multiple perimeters. By a straightforward 
generalization of Theorem~\ref{t:opg-optimal-partition} to multiple perimeters,
for an \opg instance, the length of some $S_{i,k-k'}$ must be an integer 
multiple of $\ell^*$. Similar to the single perimeter case, we can try 
all $S_{i,k-k'}$ and for each try all possible $1 \le n_{i,k-k'}^c \le n$. 
This gives us $\ell_{i,k-k'}^c = \frac{len(S_{i,k-k'})}{n_{i,k-k'}^c}$ as 
candidates for $\ell^*$; there are $n(\sum_{1\le i \le m} q_i^2)$ such 
candidates. For checking the feasibility of $\ell_{i,k-k'}^c$, we may use 
\isLFeasibleByTilingPartial($\cdot$) for the rest of $P_i$ (taking $O(q_i)$ 
time) and \isLFeasibleByTilingFull($\cdot$) for all $1 \le i' \le m$ and $i' 
\ne i$ (taking $O(\sum_{1 \le i' \le m, i' \ne i} q_{i'}^2)$ time). 
This yields a baseline algorithm that runs in 
$O(n(\sum_{1\le i \le m} q_i^2)^2)$ time. 

From here, speedups can be obtained as in the single perimeter case using 
the same reasoning. This yields a two-phase algorithm, which we call 
\algoMRG, that runs in $O((\sum_{1\le i \le m} q_i^2) \log(n + 
\sum_{1\le i \le m} q_i) + \sum_{1\le i \le m} M_i + n)$. 

%%
%The general strategy for computing $\ell^*$ remains the same: in this 
%case with multiple perimeters, some robots must still cover the optimal 
%length $\ell^*$ on some perimeter $P_i$. Following the basic reasoning, 
%each $P_i$ is checked to compute a $\ell_i^c$ that is a candidate for 
%$\ell^*$. Comparing all $\ell_i^c$ then yields $\ell^*$. For a fixed 
%$P_i$, computing $\ell_i^c$ is completed using a modified version of 
%Algorithm~\ref{algo:SRG} with the function \isLFeasibleByTilingPartial($\cdot$) 
%updated so it can handle multiple regions. 
%
%Assuming that a $P_i$ is fixed and $\ell_i^c$ is being computed. Before 
%\isLFeasibleByTilingPartial($\cdot$) is called, a candidate for $\ell_i^c$ (and 
%in turn a candidate for $\ell^*$), denoted $\ell_{i, k-k'}^c$ to include 
%an index for the perimeter, is computed. When called, 
%\isLFeasibleByTilingPartial($\cdot$) first does what it does in 
%Algorithm~\ref{algo:SRG}, which verifies that $\ell_{i, k-k'}^c$ works 
%for the rest of $P_i$ (e.g., the procedure explained with the example 
%around ~\ref{fig:tiling}). At this point, a total of $n_i^c$ robots 
%have been used to cover $P_i$, leaving $n - n_i^c$ robots to cover
%all perimeters other than $P_i$ such that no robot may cover more than
%$\ell_{i, k-k'}^c$. This can be achieved by finding the least number of 
%robots needed to cover each $P_{i'}$, $i' \ne i$ and check 
%the total is no more than $n - n_i^c$.
%
%To find the least number of robots needed to cover $P_{i'}$ such that 
%no robot covers more than $\ell_{i, k-k'}^c$, the same tiling strategy 
%is used. Unlike in the case for $P_i$, where the tiling starts from 
%the right end of $G_{i,k'}$, a left starting point for tiling is not 
%specified for $P_{i'}$. The straightforward thing to do is to try all 
%possible choices; there are $q_{i'}$ such possibilities. With this, the  
%algorithm for the general multiple perimeters case is fully specified. 
%Given the similarity of the algorithm to Algorithm~\ref{algo:SRG},
%an algorithm outline is omitted. For reference, we denote the algorithm 
%\algoMRG.
%
%To analyze the running time, we start with the updated
%\isLFeasibleByTilingPartial($\cdot$). Tiling of $P_i$ requires $O(q_i + n_i)$ 
%running time where $n_i$ is the number of robots that are needed to 
%cover $P_i$. Then, for any $P_{i'}$, $i'\ne i$, $q_{i'}$ locations on 
%$P_{i'}$ must be checked, each taking $O(q_{i'} + n_{i'})$ time, where 
%$n_{i'}$ is the number of robots needed to cover $P_{i}'$. The total time 
%spent by \isLFeasibleByTilingPartial($\cdot$) every time is then bounded by
%$O(q_i + n_i + \sum_{i' \ne i} (q_{i'}(q_{i'} + n_{i'})))$ which is 
%bounded by $O(\sum_{1 \le i \le m} q_{i}(n_{i} + q_{i}))$. Then, 
%checking all possible $k-k'$ combinations for $P_i$ (where $1 \le k, 
%k' \le q_i$) adds a multiplicative factor of $O(q_i (q_i + n_i))$. 
%Therefore, checking everything requires calling the routine
%\isLFeasibleByTilingPartial($\cdot$) about $O(\sum_{1 \le i \le m} 
%q_{i}(n_{i} + q_{i}))$ times. Similar to Algorithms~\ref{algo:MRS} 
%and~\ref{algo:SRG}, bisection may be applied here which makes the 
%total running time 
%\[
%O\big(\sum_{1 \le i \le m} \big[q_{i}(n_{i} + q_{i})\big]
%\log\big[\sum_{1 \le i \le m} q_{i}(n_{i} + q_{i})\big]\big),
%\]
%which is nearly quadratic. Again, if we consider input complexity, an
%additional $O(\sum_{1 \le i \le m} M_i)$ complexity needs to be added
%to the total. 
%
