In computing $\ell^*$ for a single perimeter $P$ with multiple connected 
components, assume that $P$ is composed of $q$ maximal connected 
components $S_1, \ldots, S_q$ (e.g., ~\ref{fig:opg-single-perimeter}),
leaving $G_1, \ldots, G_q$ as the gaps on $\partial R$. Given an 
optimal cover $\C^* = \{C_1^*, \ldots, C_n^*\}$, by 
Proposition~\ref{p:opg-endpoint-alignment}, we may assume that the left
end point of some $C_j^*$, $1 \le j \le n$ coincides with the left 
end point of some $S_k$, $1 \le k \le q$. We then look at the right 
end point of $C_j^*$. If it does not coincide with the right end point 
of some $S_{k'}$ ($k$ and $k'$ may or may not be the same), it must 
coincide with the left end point of $C_{j+1}^*$. Continuing like this, 
eventually we will hit some $C_{j'}^*$ where the right end point of 
$C_{j'}^*$ coincides with the right end point of some $S_{k'}$. 
Within a partitioned subset $C_j^*, \ldots, C_{j'}^*$, the coverage of 
each robot is minimized when $len(C_j^*) = \ldots = len(C_{j'}^*)$. 
Because $\ell^* = len(C_j^*)$ for some $1 \le j \le n$, at least one 
of the subsets must have all robots cover exactly a length of $\ell^*$. 
These two key observations are summarized as follows. 

\vspace*{-1mm}
\begin{theorem}\label{t:opg-optimal-partition}
Let $\C^* = \{C_1^*, \ldots, C_n^*\}$ be a solution to an \opg instance
with a single perimeter $P = S_1\cup \ldots\cup S_q$ and gaps $G_1, \ldots, 
G_q$. Then, $\C^*$ may be partitioned into disjoint subsets with
the following properties
\begin{enumerate}
\item the union of the individual elements from any subset forms a 
continuous line segment, 
\item the left end point of such a union coincides with the left end 
point of some $S_k$, $1 \le k \le q$, 
\item the right end point of such a union coincides with the right end 
point of some $S_{k'}$, $1 \le k' \le q$,  and
\item the respective unions of elements from any two subsets are 
disjoint, i.e., they are separated by at least one gap. 
\end{enumerate}
Moreover, for at least one such subset, $\{C_j^*, \ldots, C_{j'}^*\}$, 
it holds that $\ell^* = len(C_j^*) = \ldots = len(C_{j'}^*)$. 
\vspace*{-1mm}
\end{theorem}

\noindent\textbf{A baseline algorithm}.
In the example from ~\ref{fig:opg-example}, $\C^*$ is partitioned into 
two subsets satisfying the conditions stated in 
Theorem~\ref{t:opg-optimal-partition}. The theorem provides a way for 
computing $\ell^*$. For fixed $1 \le k, k' \le q$, denote the part of 
$\partial R$ between $S_k$ and $S_{k'}$ following a clockwise direction 
(with $S_k$ and $S_{k'}$ included) as $S_{k-k'}$.
Theorem~\ref{t:opg-optimal-partition} says that for some $k, k'$, $len(S_{k-k'})
= n_{k-k'}^*\ell^*$ for some integer $n_{k-k'}^* \in [1, n]$. We may 
find $k', k'$, and $n_{k-k'}^*$, $\ell^*$ by exhaustively going through all 
possible $k, k'$, and $n_{k-k'}^c$ (as a candidate of $n_{k-k'}^*$). For 
each combination of $k, k'$ and $n_{k-k'}^c$, we can compute a 
\begin{align}\label{eq:opg-lkkc}
\vspace*{-2mm}
\ell_{k-k'}^c = \frac{len(S_{k-k'})}{n_{k-k'}^c}
\vspace*{-2mm}
\end{align}
and check $\ell_{k-k'}^c$'s feasibility. The largest feasible 
$\ell_{k-k'}^c$ is $\ell^*$. 

For checking feasibility of a particular $\ell_{k-k'}^c$, i.e., whether 
$\ell_{k-k'}^c$ is long enough for the rest of the robots to cover the 
rest of the perimeter, we simply {\em tile} $(n - n_{k-k'}^c)$ copies 
$\ell_{k-k'}^c$ over the rest of the perimeter, starting from $S_{(k' 
\mod q) + 1}$. As an example, see ~\ref{fig:opg-tiling} where $n = 6$ 
robots are to cover the perimeter (in red, with five components $S_1, 
\ldots, S_5$). Suppose that the algorithm is currently working with 
$S_{1-2}$ (i.e., $k=1$ and $k' = 2$). If  $n_{1-2}^c = 2$, then 
$\ell_{1-2}^c = len(S_{1-2})/2$. Each of the five green line segments 
$C_1, \ldots, C_5$  in the figure has this length. As visualized in the 
figure, it is possible to cover $P\backslash S_{1-2}$ with three more 
robots, which is no more than $n - n_{1-2}^c = 4$. Therefore, this 
$\ell_{1-2}^c$ is feasible; note that it is not necessary to exhaust 
all $n = 6$ robots. In the figure, $C_3$ covers the entire $S_3$ and 
$G_3$, as well as part of $S_4$. The rest of $S_4$ is covered by $C_4$. 
As $C_4$ is tiled, it ends in the middle of $G_4$, so $C_5$ starts at 
the beginning of $S_5$. 
%
On the other hand, if $n_{1-2}^c = 3$, the resulting $\ell_{1-2}^c$ 
(each of the orange line segments has this length) is infeasible as part 
of $S_5$ is now left uncovered.
\begin{figure}[ht]
	\vspace*{-2mm}
	\begin{center}
		\begin{overpic}[width=0.6\textwidth,tics=5]{./chapters/opg/figures/tiling.eps}
			\put(13,2.7){{\small $S_1$}}
			\put(27,2.7){{\small $S_2$}}
			\put(45,2.7){{\small $S_3$}}
			\put(60,2.7){{\small $S_4$}}
			\put(86.5,2.7){{\small $S_5$}}
			\put(12.5,10){{\small $C_1$}}
			\put(25,10){{\small $C_2$}}
			\put(48,10){{\small $C_3$}}
			\put(61,10){{\small $C_4$}}
			\put(86.5,10){{\small $C_5$}}
			%\put(54,40){{\small \textcolor{green}{$P_2$}}}
			%\put(82,44){{\small $\W$}}
		\end{overpic}
	\end{center}
	\vspace*{-4.5mm}
	\caption[An illustration of the feasibility check of $\ell_{1-2}^c$]
	{\label{fig:opg-tiling}  An illustration of the feasibility check of 
		$\ell_{1-2}^c$. The single rectangular region and the perimeter (five red 
		segments $S_1$--$S_5$) are shown at the bottom. The orange and green line 
		segments show two potential	covers.}
	\vspace*{-3mm}
\end{figure}

The tiling-based feasibility check takes $O(q)$ time since there are at 
most $q$ segments to tile; it takes constant time to tile each segment 
using a given length. Let us denote this feasibility check 
\isLFeasibleByTilingPartial($k$, $k'$, $n_{k-k'}^c$), we have obtained 
an algorithm that runs in $O(nq^3)$ times since it needs to go through 
all $1 \le k \le q$, $1 \le k' \le q$, and $1 \le n_{k-k'}^c \le n$ and 
for each combination of $k, k'$, and $n_{k-k'}^c$, it makes a call to 
\isLFeasibleByTilingPartial($\cdot$). While a $O(nq^3)$ running time 
is not bad, we can do significantly better. 

\noindent\textbf{A much faster algorithm}.
In the baseline algorithm, for each $k-k'$ combination, up to $n$ 
candidate $n_{k-k'}^c$ may be attempted. To gain speedups, the first 
phase of the improved algorithm reduces the range of $\ell^*$ to limit 
the choice of $n_{k-k'}^c$. As a necessity, a feasibility check given 
only a length $\ell$ is introduced. That is, given an $\ell$, a check
is done to see whether $n$ robots are sufficient for covering $P$. 
This feasibility check is performed in a way similar to the tiling process from 
\isLFeasibleByTilingPartial($\cdot$) but now $k$ and $k'$ are not 
specified. Therefore, we need to try all $S_k$, $1 \le k \le q$ as the 
possible starting segment for the tiling. Let us denote this procedure
\isLFeasibleByTilingFull($\ell$), which runs in $O(q^2)$.

With \isLFeasibleByTilingFull($\ell$), starting from the initial bounds 
for $\ell^*$ given in Proposition~\ref{p:opg-single-bounds}, we can narrow the
bound to be arbitrarily small, using bisection, since $\ell^*$ is the 
minimum feasible $\ell$. To do this, we start with $\ell$ as the middle 
point of initial lower bound $\ell_{min}$ and upper bound $\ell_{max}$, 
and run \isLFeasibleByTilingFull($\ell$). If $\ell$ is feasible, the upper bound 
is lowered to $\ell$. Otherwise, the lower bound is raised to $\ell$. In doing 
this,  our goal in the first phase of the faster
algorithm is to reduce the range for $\ell^*$ so that there is only a 
constant number of choices for $n_{k-k'}^c$. We now derive when the 
bisection process should end. 
\setlength{\abovedisplayskip}{1.6pt}
\setlength{\belowdisplayskip}{2.0pt}
%\setlength{\abovedisplayskip}{4.2pt}
%\setlength{\belowdisplayskip}{4.2pt}

Assume that when the bisection ends, the lower and upper bounds are 
$\ell_{min}^f$ and $\ell_{max}^f$. Then, for some $\ell_{k-k'}^c$
(see~\eqref{eq:opg-lkkc}), it holds that
\[
\ell_{min}^f \le \ell_{k-k'}^c = \frac{len(S_{k-k'})}{n_{k-k'}^c} \le \ell_{max}^f.
\]
Rearranging the above yields
\[
\frac{len(S_{k-k'})}{\ell_{max}^f} \le n_{k-k'}^c \le \frac{len(S_{k-k'})}{\ell_{min}^f}.
\]
To reduce the number of possible $n_{k-k'}^c$ to at most $1$, we want 
\[
\frac{len(S_{k-k'})}{\ell_{min}^f} - \frac{len(S_{k-k'})}{\ell_{max}^f} < 1, 
\]
which is the same as requiring 
\begin{align}\label{eq:opg-lmmd}
\ell_{max}^f- \ell_{min}^f <
\frac{\ell_{max}^f\ell_{min}^f}{len(S_{k-k'})}. 
\end{align}
Noting that $\ell_{min}^f, \ell_{max}^f \ge  \ell_{min}$ and $len(S_{k-k'}) < 
len (\partial R)$, Equation~\eqref{eq:opg-lmmd} is ensured by 
\begin{align}\label{eq:opg-lmmd2}
\ell_{max}^f- \ell_{min}^f 
< \frac{\ell_{min}^2}{len(\partial R)} 
= \frac{\Big[\sum_{1\le k\le q}len(S_k)\Big]^2}{n^2len(\partial R)} 
\end{align}

Equation~\eqref{eq:opg-lmmd2} gives the stopping criteria used for refining 
the bounds for $\ell^*$. After completing the first phase, the faster algorithm
moves to the second phase of actually pinning down $\ell^*$. In this phase, 
instead of checking $\ell_{k-k'}^c$ one by one, we collect $\ell_{k-k'}^c$ for
all possible combinations of $k, k'$. Because the first phase already ensures
for each $k, k'$ combination there is at most one pair of $n_{k-k'}^c$ and 
$\ell_{k-k'}^c$, there are 
at most $q^2$ total candidates. After all candidates are collected, they are 
sorted and another bisection is performed over these sorted candidates. 
Feasibility check is done using \isLFeasibleByTilingPartial($\cdot$). The 
complete algorithm is given in Algorithm~\ref{algo:opg-SRG}. Note that 
$\ell^{min}$ and $\ell^{max}$, which change as the algorithm runs, are not 
the same as the fixed $\ell_{min}$ and $\ell_{max}$ from 
Proposition~\ref{p:opg-single-bounds}. 
% \newpage
% \vspace*{-7mm}

\begin{algorithm}[H]
	\begin{small}
		\setstretch{0.8}
		\SetKwInOut{Input}{Input}
		\SetKwInOut{Output}{Output}
		\SetKwComment{Comment}{\%}{}
		\Input{$\partial R = S_1 \cup G_1 \cup \ldots \cup S_q \cup G_q$: a 
			single boundary with the perimeter $P = S_1 \cup \ldots \cup S_q$.\\
			$n$: the number of robots}
		\Output{$\ell^*, k^*, k'^{*}$: the optimal coverage and the 
			pair of $k$ and $k'$ that realize the optimal coverage}
		\vspace{0.025in}
		
		\Comment{\footnotesize Phase one: narrow the range of $\ell^*$.}
		$\ell^{min} \leftarrow \frac{\sum_{1\le k\le q}len(S_k)}{n}$, 
		$\ell^{max} \leftarrow  \frac{len(\partial R) -  len(G_{max})}{n}$;
		\vspace{0.025in}
		
		\While{$\ell^{max}-\ell^{min} > \frac{[\sum_{1\le k\le q}len(S_k)]^2}{n^2len(\partial R)}$}{
			$\ell \leftarrow \frac{\ell^{max}+\ell^{min}}{2}$;

			\vspace{0.025in}
			(\,\isLFeasibleByTilingFull($\ell$)? $\ell^{max} \leftarrow \ell$ : 
			$\ell^{min} \leftarrow \ell$\,);

			%\uIf{\isLFeasibleByTilingFull($\ell$)}{$\ell^{max} \leftarrow \ell$;}
			%\Else{$\ell^{min} \leftarrow \ell$;}
		}
		
		\vspace{0.025in}
		\Comment{\footnotesize Phase two: pin down $\ell^*$.}
		$sm \leftarrow []$; \Comment{$sm$ is a sorted map.}
		

		\For{$k, k' \in \{1,\dots, q\}$ }{\label{algo:opg-srg:for}
			\vspace{0.025in}
			
			$n_{k-k'}^{max} \leftarrow \lfloor \frac{len(S_{k-k'})}{\ell^{min}} \rfloor$;
			$n_{k-k'}^{min} \leftarrow \lceil \frac{len(S_{k-k'})}{\ell^{max}} \rceil$;
			
			\vspace{0.025in}
			
			\For{$n_{k-k'}^c \in \{n_{k-k'}^{min}, \ldots, n_{k-k'}^{max}\}$}{\label{algo:opg-srg:for2}
				$sm$.put($\frac{len(S_{k-k'})}{n_{k-k'}^c}$, $(n_{k-k'}^c, \frac{len(S_{k-k'})}{n_{k-k'}^c}, k, k')$);
			}
		}
	
		\vspace{0.025in}
		$\ell^* \leftarrow \infty$; $k^* \leftarrow 0$; $k'^* \leftarrow 0$;
		
		\vspace{0.025in}
		\While{$sm.\mathrm{size()}$ $> 1$}{
			\Comment{\footnotesize Extract the element from sm in the middle.}
			$(n^c, \ell^c, k, k') \leftarrow sm$.middleValue(); 
	
			\vspace{0.025in}
			\uIf{\isLFeasibleByTilingPartial($k, k', n^c$)}{
				$\ell^* \leftarrow \ell^c$; $k^* \leftarrow k$; $k'^* \leftarrow k'$;
				
				$sm \leftarrow sm.$range($sm.$minKey(), $\ell^c$);
			}
			\Else{
				$sm \leftarrow sm.$removeRange($sm.$minKey(), $\ell^c$);
			}
		}		
		
		\Return{$\ell^*$, $k^*$, $k'^*$}
		\caption{\algoSRG} \label{algo:opg-SRG}
	\end{small}
\end{algorithm}
% \vspace*{-3mm}

In terms of running time, the first $\mathbf{while}$ loop starts with 
$\ell^{max} - \ell^{min} = \frac{len(\partial R) -  len(G_{max})}{n} -
\frac{\sum_{1\le k\le q}len(S_k)}{n} \le \frac{len(\partial R)}{n}$ and 
stops when $\ell^{max} - \ell^{min} \le 
\frac{[\sum_{1\le k\le q}len(S_k)]^2}{n^2len(\partial R)}$. Therefore, 
the bisection is executed 
$\log \frac{n[len(\partial R)]^2}{[\sum_{1\le k\le q}len(S_k)]^2}$ times, 
which by the assumption that $len(\partial R)$ is a polynomial factor over 
$\sum_{1\le k\le q}len(S_k)$, is $O(\log (n + q))$. Since each feasibility 
check takes $O(q^2)$ time, the first $\mathbf{while}$ loop takes 
$O(q^2\log(n + q))$ time. The $\mathbf{for}$ loops work with a total of 
$O(q^2)$ candidates and must sort them, taking time $O(q^2 \log q^2) = 
O(q^2 \log q)$. Then, the second $\mathbf{while}$ loop bisects $O(q^2)$ 
candidates and calls \isLFeasibleByTilingPartial($\cdot$) for each check, 
taking time $O(q\log q^2) = O(q\log q)$. The total running time of 
Algorithm~\ref{algo:opg-SRG} is then $O(q^2\log (n + q) + M + n)$.\footnote{We note that 
the assumption that $len(\partial R)$ is a polynomial factor over 
$\sum_{1\le k\le q}len(S_k)$ is not necessary. However, the corresponding 
analysis becomes much more involved without it. Since the assumption makes 
practical sense and also due to space limit, the more general result is 
omitted from the current chapter. Many additional interesting but 
non-essential details, including this one, will be included in an extended 
version.}
