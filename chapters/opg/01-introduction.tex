

\def\R{\mathcal R}
\def\C{\mathcal C}
\def\S{\mathcal S}
\def\P{\mathcal P}
\def\G{\mathcal G}
\def\W{\mathcal W}
%\thickmuskip=0mu
\def\opg{{\sc {OPG}}\xspace}

Consider the scenario from Fig.~\ref{fig:opg-example}, which contains a closed region with its boundary (or border) demarcated by the red and dotted blue lines. To secure the region, either from intrusions from the outside or unwanted escapes from within, it is desirable to deploy a number of autonomous agents to monitor or guard either the entire boundary or selected portions of it (e.g., the red line segments), with each agent responsible for a continuous section. 
%
Naturally, one might also want to have even coverage by the agents, e.g., minimizing the maximum effort from any agent. In practice, such efforts may correspond to sensing ranges or motion capabilities of robots, which are always limited. 
%
As an intuitive example, the figure may represent the top view of a castle, with its entire boundary being a high wall that may be traversed by agents. 
%
The portion of the wall marked with the three red line segments must be protected, whereas the art marked by the dotted blue lines may not need active monitoring (e.g., the outside may be a cliff or a body of deep water). The green and orange lines show an optimal distribution of the workload by $8$ agents that cover all red segments but skip two of the three-dotted line segments. 

More formally, in this paper, we study the problem of deploying a large 
number of robots to guard a set of one-dimensional boundary segments 
called perimeters. Each perimeter is comprised of one or more 1D segments 
that are part of a circular boundary (e.g., the red segments in 
Fig.~\ref{fig:opg-example}). Each robot is tasked to guard a continuous 1D 
segment that covers a portion of a perimeter. As the main objective, 
we seek an allocation of robots such that {\em (i)} the union of the 
robots' coverage encloses all perimeters and {\em (ii)} the 
maximum coverage of any robot is minimized. We call this 1D deployment
problem the Optimal Perimeter Guarding (\opg) problem. 
\begin{figure}[ht]
\vspace*{0mm}
\begin{center}
\begin{overpic}[width=0.7\textwidth,tics=5]{./chapters/opg/figures/example.eps}
%\put(26,20){{\small $R_1$}}
%\put(20,39){{\small \textcolor{red}{$P_1$}}}
%\put(66,28){{\small $R_2$}}
%\put(54,40){{\small \textcolor{green}{$P_2$}}}
%\put(82,44){{\small $\W$}}
\end{overpic}
\end{center}
\vspace*{-5mm}
\caption{\label{fig:opg-example} An illustrative scenario where a perimeter, 
in this case represented as the red line segments, must be guarded by 
$n = 8$ robots, which are constrained to 
only travel along the perimeter boundary (the red line segments plus the 
dotted blue lines, which are gaps that do not need to be guarded). An 
optimal set of locations for the $8$ robots and the coverage region for 
each robot are marked on the perimeter boundary in green and orange, 
which minimizes the maximum coverage required for any robot.}
\vspace*{-8mm}
\end{figure}


In this work, three main \opg variants are examined. The settings 
regarding the perimeter in these three variants are: {\em (i)}
multiple perimeters with each having a single connected component; 
{\em (ii)} a single perimeter containing multiple connected components; 
and {\em (iii)} multiple perimeters with each containing multiple 
connected components (the most general case). For all three variants, 
we have developed exact algorithms for solving \opg that runs in low
polynomial time. More specifically, let there be $n$ robots, $m$ 
perimeters, with perimeter $i$ ($1 \le i \le m$) containing $q_i$ 
connected components. If $m = 1$, then let the only perimeter contains
$q$ connected components. For the three variants, our algorithm 
computes an optimal solution in time $O(m(\log n + \log m) + n)$, 
$O(q^2\log(n+q) + n)$, and $O((\sum_{1\le i \le m} q_i^2) \log(n + 
\sum_{1\le i \le m} q_i) + n)$, respectively, which
are roughly quadratic in the worst case. 
In addition to computing locations for deploying the robots, we 
further compute shortest paths for deploying the robots, given some 
initial configuration of the robots. The modeling of the \opg 
problem and the development of the efficient algorithms for \opg 
constitute the main contribution of this paper. 

With an emphasis on robotic swarm deployment, within multi-robot 
systems research \cite{arai2002advances,gerkey2004formal,ren2008distributed,bullo2009distributed}, 
our study is closely related to {\em formation control}, e.g., 
\cite{ando1999distributed,jadbabaie2003coordination,olfati2004consensus,ren2005consensus,cheng2008almost,mesbahi2010graph,yu2012rendezvous},
where the goal is to achieve certain {\em distributions} through 
continuous (often, local sensing based) interactions among the 
agents or robots. Depending on the particular setting, the 
distribution in question may be spatial, e.g., rendezvous
\cite{ando1999distributed,yu2012rendezvous}, or maybe an agreement 
in agent velocity is sought \cite{jadbabaie2003coordination,ren2005consensus}. 
In these studies, the resulting formation often have some 
degree-of-freedoms left unspecified. For example, rendezvous 
results \cite{ando1999distributed,yu2012rendezvous} often come 
with exponential convergence guarantee, but the location of
rendezvous is generally unknown {\em a priori}. 

On the other hand, in multi-robot task and motion planning problems (e.g.,
\cite{smith2009monotonic,ayanian2010decentralized,liu2011multi,liu2013optimal,turpin2014goal,turpin2014capt,alonso2015multi,SolYu15}), 
especially ones with a {\em task allocation} element 
\cite{smith2009monotonic,liu2011multi,liu2013optimal,turpin2014goal,turpin2014capt,SolYu15},
the (permutation-invariant) target configuration is often mostly 
known. The goal here is finding a one-to-one mapping between individual 
robots and the target locations (e.g., deciding a {\em matching}) and 
then plan (possibly collision-free) trajectories for the robots to reach 
their respective assigned targets \cite{turpin2014goal,turpin2014capt,SolYu15}.  
In contrast to formation control and multi-robot motion planning research, 
our study of \opg seeks to determine an exact, optimal distribution 
pattern of robots (in this case, over a fairly arbitrary, bounded 1D 
topological domain). Thus, solutions to \opg may serve as the target 
distributions for multi-robot task and motion planning, which is a main 
motivation behind our work. The generated distribution pattern is 
also potentially useful in multi-robot persistent monitoring 
\cite{soltero2014decentralized} and coverage \cite{howard2002mobile,schwager2009optimal} 
applications, where robots are asked to carry out sensing tasks in some 
optimal manner. 
	
As a multi-robot coverage problem, \opg is intimately connected to Art 
Gallery problems \cite{o1987art,shermer1992recent}, with origins traceable 
to half a century ago \cite{klee1969every}. Art Gallery problems assume 
a visibility-based \cite{lozano1979algorithm} sensing model; in a typical 
setup \cite{o1987art}, the {\em interior} of a polygon must be visible to at 
least one of the guards, which may be placed on the boundaries, corners, 
or the interior of the polygon. Finding the optimal number of guards are 
often NP-hard \cite{lee1986computational}. Alternatively, disc-based sensing 
model may be used, which leads to the classical {\em packing} problem 
\cite{thue1910dichteste,hales2005proof}, where no overlap is allowed between 
the sensors' coverage area, the {\em coverage} problem 
\cite{drezner1995facility,cortes2004coverage,pavone2009equitable,martinez2010distributed,pierson2017adapting}, 
where all 
workspace must be covered with overlaps allowed, or the {\em tiling} problem 
\cite{kershner1968paving}, where the goal is to have the union of sensing
ranges span the entire workspace without overlap. For a more complete 
account on Art Gallery, packing, and covering, see Chapters 2, 3, and 33 of
\cite{toth2017handbook}. Despite the existence of a large body of literature 
performing extensive studies on these intriguing computational geometry 
problems, these types of research mostly address domains that are 2D and 
higher. To our knowledge, \opg, as an optimal coverage problem over 
a non-trivial 1D topological space, represents a practical and novel formulation 
yet to be fully investigated. 

The rest of the section is organized as follows. 
%
The \opg problem and some of its most basic properties are described 
in Section~\ref{section:opg-problem}. 
%
In Section~\ref{section:opg-analysis}, a thorough structural analysis of \opg 
with single and multiple perimeters is performed, paving the way for 
introducing the full algorithmic solutions in Section~\ref{section:opg-algorithm}.
%
Then, in Section~\ref{section:opg-evaluation}, comprehensive numerical 
evaluations of the multiple polynomial-time algorithms are carried out. 
In addition, two realistic application scenarios are demonstrated. 
%
In Section~\ref{section:opg-conclusion}, we conclude with  
additional discussions. 
