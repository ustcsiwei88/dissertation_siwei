In designing efficient algorithms, the solution structure of \opg induced by 
the problem formulation is first explored, starting from the case where there 
is a single region.

\subsubsection{Guarding a Single Region}
\noindent\textbf{Perimeter with a single connected component}. For 
guarding a single region $\R = \{R\}$, i.e., there is a single 
boundary $\partial R$ to be guarded, all $n$ robots can be directly 
allocated to $\partial R$. If the single perimeter $P \subset 
\partial R$ further has a single connected component that is either 
homeomorphic to $\mathbb S^1$ or $[0, 1]$, then each robot $j$ can 
be assigned a piece $C_j \subset P$ such that $\bigcup_{C_j \in \C} 
C_j = P$ and $len(C_j) = len(P)/n$. Clearly, such a cover $\C$ is 
also an optimal cover. 

\noindent\textbf{Perimeter with multiple maximal connected components}. 
When there are multiple maximal connected components (or segments) in a 
single perimeter $P$, things become more complex. To facilitate the 
discussion, assume here $P$ has $q$ segments $S_1, \ldots, S_q$ arranged 
in the clockwise direction (i.e., $P = S_1 \bigcup \ldots \bigcup S_q$), 
which leaves $q$ gaps $G_1, \ldots, G_q$ with $G_k$ immediately following 
$S_k$. ~\ref{fig:opg-single-perimeter} shows a perimeter with five segments
and five gaps. 
\begin{figure}[ht]
\vspace*{-1mm}
\begin{center}
\begin{overpic}[width=0.7\textwidth,tics=5]
{./chapters/opg/figures/single-perimeter.eps}
\put(16,7){{\small $S_1$}}
\put(17,31){{\small $G_1$}}
\put(46,30){{\small $S_2$}}
\put(56.5,28){{\small $G_2$}}
\put(72,27.5){{\small $S_3$}}
\put(84.5,17.5){{\small $G_3$}}
\put(85,6){{\small $S_4$}}
\put(72,-4){{\small $G_4$}}
\put(59,0){{\small $S_5$}}
\put(45,4){{\small $G_5$}}
\end{overpic}
\end{center}
\vspace*{-1mm}
\caption{\label{fig:opg-single-perimeter} A perimeter $P$ with five segments 
$S_1, \ldots, S_5$, separated by five {\em gaps} $G_1, \ldots, G_5$.}
\vspace*{-3mm}
\end{figure}

Suppose an optimal set of assignments for the $n$ robots guarding $P$ 
and  satisfying~\eqref{eq:opg-coverage} and~\eqref{eq:opg-objective} is $\C^* 
= \{C_j^*\}$. Let $G_{max}$ be a largest gap, i.e., $len(G_{max}) = 
\max_{1 \le k \le q}len(G_k)$. Via small perturbations to the lengths 
of $G_k$, we may also assume that $G_{max}$ is unique. If we exclude 
the gap $G_{max}$ and have the $n$ robots cover the rest of $\partial 
R$ evenly, then it must hold that $len(C_j^*) \le (len(\partial R) - 
len(G_{max}))/n$. On the other hand, $len(C_j^*) \ge (\sum_{1\le k\le 
q}len(S_k))/n$ always holds. These yield a pair of basic upper and 
lower bounds for the optimal single robot coverage length $\ell^*$, 
summarized as follows. 

\begin{proposition}\label{p:opg-single-bounds} Define 
\[
\ell_{min} = \frac{\sum_{1\le k\le q}len(S_k)}{n}\,\,and\,\,
\ell_{max} = \frac{len(\partial R) -  len(G_{max})}{n},
\]
it holds that 
\begin{align}\label{eq:opg-lopt} 
\ell_{min} \le \ell^* \le \ell_{max}.
\end{align}
\end{proposition}

On the other hand, it is not always the case that a largest gap $G_{max}$, 
even if unique, will be skipped by $\bigcup_{C_j \in \C^*} C_j^*$. 
That is, an optimal cover $C^*$ may enclose the largest gap. 
 
\begin{proposition}\label{p:opg-max-no-exclusion}
Given a region $R$ and perimeter $P \subset \partial R$, let
$G_{max}$ be the unique longest connected component of $\partial R
\backslash P$. Let $\C^*$ be an optimal cover of $P$. Then, it is 
possible that $G_{max} \subset C_j^*$ for some $C_j^* \in \C^*$. 
\end{proposition}
\begin{proof}The claim may be proved via contradiction with the example 
illustrated  in ~\ref{fig:opg-max-not-skipped} which readily generalizes. 
In the figure, there are four gaps $G_1, \ldots, G_4$, in which three 
gaps ($G_1$, $G_2$, and $G_4$) have the same length (i.e., $len(G_1)= 
len(G_2)= len(G_4)$) and are evenly spaced (i.e., $len(S_1)= len(S_2) 
= len(S_3\cup G_3\cup S_4)$). Here, $G_{max} = G_3$, which is 
$1.5$ times the length of other gaps, i.e., $len(G_3) = 
\frac{3}{2}len(G_1)$. 
\begin{figure}[ht]
\vspace*{0mm}
\begin{center}
\begin{overpic}[width=0.7\textwidth,tics=5]{./chapters/opg/figures/max-not-skipped.eps}
\put(27,34){{\small $S_1$}}
\put(48,44){{\small $G_1$}}
\put(69,34){{\small $S_2$}}
\put(69,9){{\small $G_2$}}
\put(62,1){{\small $S_3$}}
\put(48,-4){{\small $G_3$}}
\put(32,2){{\small $S_4$}}
\put(26,8){{\small $G_4$}}
\end{overpic}
\end{center}
\vspace*{-3mm}
\caption{\label{fig:max-not-skipped} A case where the perimeter has 
four segments or maximal connected components. Three of the gaps, 
$G_1$, $G_2$, and $G_4$ are of the same length and are evenly spaced, 
$G_3$ is $0.5$ times longer.}
\vspace*{-3mm}
\end{figure}

For $n = 3$ robots, the optimal cover $\C^*$ must allocate each robot 
to guard each of $S_1$, $S_2$, and $(S_3\cup G_3\cup S_4)$. Without 
loss of generality, let $C_1^* = S_1$, $C_2^* = S_2$, and $C_3^* = 
(S_3\cup G_3\cup S_4)$. This means that $G_3$ is covered by $C_3^*$ 
and not skipped by $\C^*$. In this case, $len(C_1^*) =len(C_2^*) = 
len(C_3^*) = len (S_1)$.

To see that this must be the case, suppose on the contrary that $G_3$ 
is skipped and let $\C = \{C_1, C_2, C_3\}$ be an alternative cover. 
By Lemma~\ref{l:opg-no-partial-coverage}, an optimal cover must skip $G_3$ 
entirely. In this case, some $C_j$, say $C_1$, must have its left end 
point\footnote{In this paper, for a non-circular segment or gap, its 
left end point is defined as the {\em limit point} along the 
counterclockwise direction along the perimeter and its right end point 
is defined as the limit point in the clockwise direction along the 
perimeter. So, in ~\ref{fig:opg-max-not-skipped}, for $S_1$, its left 
end point touches $G_4$ and its right end point touches $G_1$.} coincide 
with the right end point of of $G_3$ (the point where $G_3$ meets $S_4$). 
Then $C_1$ must cover $S_4$ and $G_4$; otherwise, $C_2$ and $C_3$ must 
cover $S_1 \cup S_2 \cup S_3$, which makes $len(C_2) + len(C_3) \ge 
len(S_1 \cup S_2 \cup S_3) > 2len(S_1)$ and $\C$ a worse cover than 
$\C^*$. By symmetry, similarly, some $C_j$, say $C_3$, must have its 
right end point coincide with the left end point of $G_3$ and cover 
$S_3$ and $G_2$. However, this means that both $G_2$ and $G_4$ are 
covered by $\C$. Even if $G_1$ is skipped, this makes $len(C_1 \cup 
C_2 \cup C_3) = len(S_4 \cup G_4 \cup S_1 \cup S_2 \cup G_2 \cup S_3) 
> len (S_1 \cup S_2 \cup S_3\cup G_3\cup S_4) > 3len(S_1)$, again making
$\C$ sub-optimal. By the pigeonhole principle, at least one of the $C_1$, 
$C_2$, or $C_3$ must be longer than $len(S_1)$. Therefore, skipping 
$G_{max} = G_3$ in this case leads to a sub-optimal cover. The only 
other alternative for an optimal cover with $n = 3$ is to have $C^* = 
\{S_1, S_2, (S_3\cup G_3\cup S_4)\}$. 
\end{proof}

Proposition~\ref{p:opg-max-no-exclusion} implies that in allocating robots 
to guard a perimeter $P \subset \partial R$, an algorithm cannot simply 
start by excluding the longest component from $\partial R 
\backslash P$ and then the next largest, and so on. This makes solving
\opg more challenging. Referring back to ~\ref{fig:opg-example}, if the 
top gap is skipped by the cover, then the three robots on the right side 
of the perimeter (two orange and one green) need to cover the part of the 
perimeter between the two hexagons. This will cause $\ell^*$ to increase. 

On the other hand, for an optimal cover $\C^* = \{C_1^*, \ldots, C_n^*\}$ 
of $P$, some $C_j^*\in \C^*$ must have at least one of its end point 
aligned with an endpoint of a component $S_k$ of $P$ (assuming that $P 
\subsetneq \partial R$). 

\begin{proposition}\label{p:opg-endpoint-alignment}
For an optimal cover $\C^* = \{C_1^*,\ldots, C_n^*\}$ of a perimeter 
$P = S_1 \bigcup \ldots \bigcup S_q \subset \partial R = S_1 \bigcup G_1 
\bigcup $ $\ldots \bigcup S_q \bigcup G_q$, for some $S_i \subset P$ and 
$C_j^* \in C^*$, their right (or left) endpoints must coincide. 
\end{proposition}
\begin{proof}
By Lemma~\ref{l:opg-no-partial-coverage}, for any $G_k \subset \partial R 
\backslash P$, and $C_j^* \in \C^*$, $G_k \cap C_j^* = G_k$ or 
$G_k \cap C_j^* = \emptyset$. Since at least one $G_k$, $1 \le k \le q$, 
must be skipped by $C_1^* \cup \ldots C_n^*$, some $C_j^*$, $1 \le j \le n$ 
must have its right endpoint aligned with the right endpoint of $S_k$, 
which is on the left of $G_k$. Following the same argument, some 
$C_{j'}^*$ and $S_{k'}$ must have the same left endpoints. 
\end{proof}

Proposition~\ref{p:opg-endpoint-alignment} suggests that we may attempt to cover 
a perimeter $P$ starting from an endpoint of $S_1$, $S_2$, and so on. 
%An algorithm can attempt a binary search approach to first test whether 
%$\ell^* \le \frac{\ell_{min} + \ell_{max}}{2}$ or $\ell^* > \frac{\ell_{min} 
%+ \ell_{max}}{2}$. For each candidate of $\ell^*$, we need to try at most $q$ 
%left starting points for $C_1$. This yields an algorithm that allows the 
%approximation of $\ell^*$ arbitrarily well for a given $P$ and $n$.  
Indeed, as we will show in Section~\ref{section:opg-algorithm}, an efficient 
algorithm can be designed exploiting this important fact. 


\subsubsection{Guarding Multiple Regions}
In a multiple region setup, there is one additional level of complexity:
the number of robots that will be assigned to an individual region is 
no longer fixed. This introduces another set of variables $n_1, \ldots, 
n_m$ with $n_1 + \ldots + n_m = n$ and $n_i$, $1 \le i \le m$ being the 
number of robots allocated to guard $\partial R_i$. For a fixed $n_i$, 
the results derived for a single region, i.e., 
Propositions~\ref{p:opg-single-bounds}--\ref{p:opg-endpoint-alignment} continue 
to hold.

