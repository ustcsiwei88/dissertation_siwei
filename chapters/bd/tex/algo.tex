In this section, we describe three methods for solving infinite-horizon \prob (Problem~\ref{prob:bd-1}).
Sec.~\ref{sec:bd-dp} provide a dynamic programming (DP) algorithm based on one developed in \cite{adler2022role}, 
followed by Sec.~\ref{sec:bd-ilp} formulating an integer linear programming model,
and Sec.~\ref{sec:bd-dp_local} discusses a heuristic search algorithm based on the DP algorithm. 
The extension to an infinite attack stream with a finite look-ahead horizon  (Problem~\ref{prob:bd-2}) will be discussed in Sec.~\ref{sec:bd-hor}.


\subsection{Exact Dynamic Programming Based Method}%for A Fixed Number of Defenders}
\label{sec:bd-dp}

The DP algorithm builds a recursion formula on the last attacks intercepted by the defenders.
Without loss of generality, we assume that each attack is only intercepted by one defender.
For a DP state $a_1, \dots, a_k$ where $a_i$ is the last attack defender $i$ intercepts, 
let $T[a_1]\dots[a_k]$ store the maximum number of attacks that can be intercepted when the $i^{th}$ defender's last intercepted attack events is $a_i$ $(i\in[1,k])$,
and denote $a_{ma}$ as the maximum of $a_1, \dots, a_k$.
Base on which attack the $ma^{th}$ defender intercepts before intercepting attack $a_{ma}$,
we can write the DP recursion formula as follows.
\begin{equation}
\begin{split}
T[a_1]\dots[a_{ma}]\dots[a_k] &= \\ 
\max_{p\in prev(a_{ma}, ma) \wedge p\neq a_1 \dots a_k} &  T[a_1]\dots[p]\dots[a_k] + 1
\end{split}
\end{equation}

Pseudo-code in Alg.~\ref{alg:bd-dp} provides a sketch of a possible implementation of the dynamic programming algorithm.
Effectively implemented, the time complexity of running Alg.~\ref{alg:bd-dp} is $O( (n+1)^{k+1})$, which is polynomial when $k$, the number of defenders, is fixed.
\vspace{-2mm}

\begin{algorithm}
%\begin{small}
\DontPrintSemicolon
\KwData{
$E=\big \langle loc_i, t_i\big\rangle_{i=1}^{n}$: $n$ attack events\;
$loc^1,\dots,loc^k$: initial locations of the $k$ defenders\;
$v_1,\dots,v_k$: speeds of the $k$ defenders\;
}
\KwResult{Maximum number of attacks intercepted}
\vspace{1mm}
$T\gets$ an $(n+1)^k$-length array initialized to $-\infty$\;
\vspace{1mm}
$result\gets 0$\;
\vspace{1mm}
$T[0]\gets 0$\;
\vspace{1mm}
\For{$mask\gets 0 $ \KwTo $(n+1)^k - 1$}{
\vspace{1mm}
    $\overline{a_1 a_2\dots a_k} \leftarrow mask$\;
    \Comment{$\overline{a_1 a_2\dots a_k}$ represents a base-($n+1$) number, i.e., $a_1\cdot (n+1)^{k-1} + a_2 \cdot (n+1)^{k-2} +\dots+ a_k$}
    \If{$\exists\ a_i = a_j$}{
        \Continue\;
    }
\vspace{1mm}
    $ma \leftarrow argmax_i a_i$\;
\vspace{1mm}
    \For{$p\in prev(a_{ma}, ma)$}{
        \If{$\forall i\ pos_i \neq p$}{
            $pm\gets mask - (a_{ma} - p)\cdot (n+1)^{k-ma}$\;
            \Comment{$pm$ is the result of replacing $a_{ma}$ with $p$}
            $T[mask]\gets \max(T[mask], T[pm] + 1)$\;
        }
    }
\vspace{1mm}
    $result\gets \max(result, T[mask])$\;
}
\vspace{1mm}
\Return{$result$}\;

\caption{Dynamic Programming for \prob}
\label{alg:bd-dp}
%\end{small}
\end{algorithm}
\vspace{-2mm}
%\vspace{-15mm}

The algorithm presented here is a slight modification of the DP algorithm in \cite{adler2022role} with two subtle differences. First, we enforce that an attack event can only be handled by one defender.
Second, we explicitly use the initial locations of the defenders in the algorithm, which is essential for handling the finite horizon extension.

\subsection{Solving \prob with Integer Linear Programming Model based on a Flow Formulation}
\label{sec:bd-ilp}

It is not difficult to see that \prob can also be seen as a network flow problem by treating each attack event as a node and the reachability between each pair of attack events for each defender as the edges in the graph. 

Specifically, there are $n$ nodes in the graph, representing the $n$ attack events. 
There are also $O(n^2k)$ connections between nodes inside the graph.
If defender $j$ can reach attack event $i'$ from $i$, there is an connection 
$edge[i][i'][j]$ between node $i$ and $i'$. 
Also, we use a binary variable $intercept[i]$ to denote whether attack event $i$ is successfully intercepted. 
These give rise to the following integer linear programming (ILP) formulation of the problem.

% \begin{gather}
Eq. \eqref{eq:bd-intercept} sets the criteria of an attack $i$ being intercepted as at least one defender to come into the node $i$, which means intercept attack $i$.
Eq. \eqref{eq:bd-flow} sets the defender flow conservation rule that the number of type $j$ defender exiting node $i$ must be larger than or equal to the number of type $j$ defender coming to node $i$.
Eq. \eqref{eq:bd-initial} sets the initial constraints on the number of each type of defender used (coming out from node 0).

\begin{gather}
\sum_{j\in[1, k],\ i'\in prev(i, j)} edge[i'][i][j] \geq intercept[i] \label{eq:bd-intercept}\\
\sum_{nxt_i\in next(i, j)} edge[i][nxt_i][j] \leq \sum_{prv_i \in prev(i,j)} edge[prv_i][i][j]  \label{eq:bd-flow} \\
\sum_{nxt_0 \in next(0, j)} edge[0[nxt_0][j] \leq 1 \label{eq:bd-initial}  \\
Objective\quad \max \sum_i intercept[i]
\end{gather}

% \end{gather}
Denote $M$ as the number of connections in the graph, clearly $M<n^2k$.
This integer linear programming formulation uses $M + n$ variables, and 
$nk + n$ constraints. 

%It may not be worth mentioning
\begin{remark}
The flow formulation of the problem is an NP-hard problem. The proof is similar to the NP-completeness proof of the two-commodity flow in \cite{even1975complexity}. This seems to suggest that \prob is NP-hard as well. 
\end{remark}

% \begin{proof}
% \end{proof}

\subsection{Exhaustive Defenders Pairing Heuristic Search Method}
\label{sec:bd-dp_local}
We now develop a heuristic search method using the dynamic programming algorithm discussed in Alg.~\ref{alg:bd-dp} 
applied on two defenders.
The DP algorithm that computes the optimal solution for two defenders is used as a local improvement primitive 
for the local search heuristic algorithm. 

We call the resulting algorithm \emph{exhaustive defender pairing} (\ours).  
In each iteration, \ours pick two defenders, and the attack events the two defenders have intercepted and the attack events that have not been intercepted by any defender. 
Then, \ours uses the DP algorithm in Alg.~\ref{alg:bd-dp} to increase the number of attack events intercepted by the two defenders selected. 
The complete algorithm is sketched in Alg.~\ref{alg:bd-dp_local}.

\begin{algorithm}[h]
\DontPrintSemicolon
\KwData{
$E=\big \langle loc_i, t_i\big\rangle_{i=1}^{n}$: $n$ attack events\;
$loc^1,\dots,loc^k$: initial locations of the $k$ defenders\;
$v_1,\dots,v_k$: speeds of the $k$ defenders\;
}
\KwResult{Number of attacks intercepted}

$Intercept \gets$ a length-$n$ array initialized to $-1$\;
\Comment{$Intercept$ array stores for each event the defender that intercepts it}
$result\gets 0$\;
\vspace{1mm}
\For{$u, v \in \{1, \dots, k\}\times\{1, \dots, k\}, u\neq v$}{
    $E'\gets \{w\ |\ Intercept[w] \in\{u, v, -1\}\}$ \;
    \Comment{$E'$ stores the set of attack events intercepted by defender $u,v$ and the attacks not intercepted by any defender}
\vspace{1mm}
    $\tilde{n}\gets E'.size$\;
\vspace{1mm}
    T $\gets$ an $(\tilde{n}+1)^2$-length array initialized to $-\infty$\;
\vspace{1mm}
    $T[0]\gets 0$\;
    \Comment{Apply the DP algorithm for defender $u$ and $v$}
\vspace{1mm}
    \For{$mask\gets 0 $ \KwTo $(\tilde{n}+1)^2-1$}{
        $\overline{a_1 a_2} \leftarrow mask$\;
        \If{$a_1 =a_2$}{
            \Continue\;
        }
\vspace{1mm}
        $ma \leftarrow argmax_i a_i$\;
\vspace{1mm}
        \For{$p\in prev(a_{ma}, ma)$}{
            \If{$\forall i\ a_i \neq p$}{
                $pm\gets mask - (a_{ma} - p)\cdot (n+1)^{2-ma}$\;
                $T[mask]\gets \max(T[mask], T[pm] + 1)$\;
            }
        }
        % $result\gets \max(result, T[mask])$\;
    }
\vspace{1mm}
    \If{Solution is improved}{
        Update $Intercept, result$\;
    }
}
\vspace{1mm}
\Return{$result$}
\caption{Exhaustive Defender Pairing}
\label{alg:bd-dp_local}
\end{algorithm}

We can try different defender pairing orders and choose the best one. 
In our \ours implementation, we choose to run line 3 of Alg.~\ref{alg:bd-dp_local} in 3 different iteration ordering of $u, v$.
Alg.~\ref{alg:bd-dp_local}'s running time is $O(k^2 n^3)$ as we try $O(k^2)$ pairs of defenders, and each run of the DP algorithm takes $O(n^3)$.
% ($u$ increasing from $1$ to $k$, decreasing from $k$ to $1$, increasing from $k/2$ to $k$ after which go back to $1$ and increase to $k/2-1$). 

\subsection{Handling Infinite Attack Streams with a Finite Look-Ahead Horizon }
\label{sec:bd-hor}
For Problem~\ref{prob:bd-2} where the attacks $\big \langle loc_i, t_i\big \rangle_{i=1}^{\infty}$ may be infinite, and the look-ahead horizon is finite, not all attacks are revealed at once. As such, previous methods cannot be directly applied. 
As the future attack sequence cannot be foreseen, defenders need to \emph{react} to information (e.g., the attack events in the next $T$ time interval) obtained so far. 

Towards addressing the problem, a greedy \emph{replanning} approach can be applied. Whenever an attack event is observed, the defender team replans the capture sequence given the new attacks added to the attack queue. 
This gives rise to an online algorithm sketched in Alg.~\ref{alg:bd-horizon}.

\begin{algorithm}[h]
\DontPrintSemicolon
\KwData{
$E=\big \langle loc_i, t_i\big\rangle_{i=1}^{\infty}$: a stream of attack events\;
$loc^1,\dots,loc^k$: initial locations of the $k$ defenders\;
$v_1,\dots,v_k$: speeds of the $k$ defenders\;
}
$E'\gets $ an empty queue\;
\Comment{$E'$ stores attack events seen so far}
\vspace{1mm}
\While{new attack events added to $E'$ }{
    Apply Alg.~\ref{alg:bd-dp_local} to compute a plan for the defenders and attack events $E'$\;
    Execute the plan, pop out from $E'$ attack events passed, and update $loc^1,\dots,loc^k$ to the defenders' current locations until new attacks are foreseen\;
}
% \KwResult{Number of attacks intercepted}

% $Intercept \gets$ a length-$n$ array initialized to $-1$\;
% \Comment{$Intercept$ array stores for each event the defender that intercepts it}
% $result\gets 0$\;
% \For{$u, v \in \{1, \dots, k\}\times\{1, \dots, k\}, u\neq v$}{
%     $E'\gets \{w\ |\ Intercept[w] \in\{u, v, -1\}\}$ \;
%     \Comment{$E'$ stores the set of attack events intercepted by defender $u,v$ and the attacks not intercepted by any defender}
%     $\tilde{n}\gets E'.size$\;
%     T $\gets$ an $(\tilde{n}+1)^2$-length array initialized to $-\infty$\;
%     $T[0]\gets 0$\;
%     \For{$mask\gets 0 $ \KwTo $(\tilde{n}+1)^2-1$}{
%         $\overline{a_1 a_2} \leftarrow mask$\;
%         \If{$pos_0 = pos_1$}{
%             \Continue\;
%         }
%         $ma \leftarrow argmax_i pos_i$\;
%         \For{$p\in prev(a_{ma}, ma)$}{
%             \If{$\forall i\ pos_i \neq p$}{
%                 $T[mask]\gets \max(T[mask], 1 + T[\overline{a_1\dots p \dots a_k}])$\;
%             }
%         }
%         % $result\gets \max(result, T[mask])$\;
%     }
%     \If{Solution is improved}{
%         Update $Intercept, result$\;
%     }
% }
% \Return{$result$}
\caption{Online Exhaustive Defender Pairing}
\label{alg:bd-horizon}
\end{algorithm}