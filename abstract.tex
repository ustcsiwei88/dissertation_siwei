\begin{my_abstract}

Robots or robotic applications, like living beings, perceive the world through
a great varieties of sensors.
For range sensors, be it acoustic like sonar, photic like cameras, laser beams or lidars, 
the sensing range can, to some extent, be ideally abstracted as either 1D lines or 2D circles.
Hence, it would be helpful to study the properties of the underlying geometric problems.

% including 
% how to deploy line-of-sight sensors to guard the perimeter of some region such that intruders cannot penetrate through, 
% how to configure a sensor network with mobile range sensors to fully cover a region,
% and how to arrange barriers such that different sets of regions can be separated from each other.

The dissertation starts with a general review of different sorts of range sensors, 
especially range and line sensors used in real world applications. 
Then, we will introduce the general overview of the mathematical background and tools used for this thesis. 
The main body of this dissertation is divided into five parts, each of them comes with 
a specific research problem. 

The first chapter talks about perimeter guarding with 1D sensors, where the sensing range 
is assumed to be a continous line on top of the perimeter of some regions. Two problems are 
introduced, perimeter guarding with homogeneous sensors and with heterogeneous sensors. 
The homogeneous case can be solve using only classical algorithms. 
While the heterogeneous case is NP-hard, but can still be solved with dynamic programming in time. 

The second chapter continues with perimeter guarding but uses circles to represent the 
coverage range of sensors. The study extends to guarding 2D regions beyond the perimeters. 
Our study turns out that even for covering the boundary of a simple polygon, 
the problem of find the minimum sensing radius is NP-hard to approximate within a factor of 1.152. 
\todo{experiment on the minimum number of circles? and the comparison with classical methods like k-means and new methods like immitation learning? 
Guess it can result in a journal? And typos in the previous RSS 2020 paper looks serious.} 
However, effective approximation algorithms are developed to solve this problem. 

After the two chapters talking about covering perimeters or regions, which is essentially separating 
some crtical polygonal regions from the outside. 
The third chapter digs deeper into this problem by studying the separation of more than two polygonal sets. 
To simplify the problem, a line-of-sight sensing model will be adopted. 
Then problem is NP-hard even for the problem of separating two sets of regions with the minimum number of lines.
Still, a near-optimal solution using integer programming is provided.

The fourth chapter duels with the dynamical setting, two different problems but related are studied. 
the first problem is the boundary defense problem in the context of heterogeneous defenders, which is an extension of the previous
perimeter defense probelm. 
The second problem is the coordinated sweeping problem where a group of robots coordinated to sweep a region with obstacles. 

The last chapter goes into reality by introducing two relevant real-world applications. 
The first one is the placement of UV (ultraviolat) lights for the sanitization purposes.
The second one is the placement of tracking cameras or surveilance cameras. 
\todo{the placement of tracking cameras have not resulted in a paper yet, though the experiment and data is there} 

% \begin{itemize}
%     \item Perimeter guarding with homogeneous or heterogeneous 1D line-of-sight sensors
%     \item Covering the boundary or the interior of a polygon with range sensors
%     \item Separating polygonal sets with the minimum number of barriers
%     \item Dynamic sweep line coverage and perimeter defense
%     \item Relevant applications
% \end{itemize}

\end{my_abstract}
